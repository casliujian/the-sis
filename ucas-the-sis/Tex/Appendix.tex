\chapter{\sctlprov{}的输入语言描述}\label{chapt:sctl:input:language}

\section{词法标记与关键字}
输入文件中包含的是有限个依次排列的字符,这些字符通过词法解析器的解析生成有限个词法标记。在词法标记中,
\begin{itemize}
	\item 一个非负整数表示为有限个连续数字,而一个负整数则表示为在一个非负整数前添加符号“\code{-}”;
	\begin{center}
		\code{integer ::= [-]\{0...9\}+}
	\end{center}
	\item 一个非负浮点数表示为以符号“\code{.}”分隔的两个非负整数,其中当“\code{.}”后边的非负整数为\code{0}时可省略,而一个负浮点数则表示为在一个非负浮点数前添加符号“\code{-}”;
	\begin{center}
		\code{float	::= [-]\{0...9\}+.\{0...9\}*}
	\end{center}
	\item 一个标识符表示为以字母开头,并且有限个字母、数字、下划线以及符号“\code{-}”的组合,而且,在输入语言中区分以大写字母开头的标识符和以小写字母开头的标识符,不同标识符的详细用法可参考下面具体的语法描述。
	\begin{center}
		\begin{tabular}{lll}
			\code{iden}&	\code{::=} &  \code{letter \{letter|uletter|0...9|\_|-\}*}\\
			\code{uiden} & \code{::=}&  \code{uletter \{letter|uletter|0...9|\_|-\}*}\\
			\code{letter} &	\code{::=} &  \code{a...z}\\
			\code{uletter} & \code{::=} & \code{A...Z}
		\end{tabular}	
	\end{center}
	\item 空白符和注释被用来增强程序的可读性,且在输入文件的语法分析阶段被忽略掉。其中,空格、制表符、回车符、换行符均为空白符。注释分为单行注释与多行注释,其中单行注释以“\code{//}”开头且以换行符结尾,而多行注释分为两种:以“\code{/*}”开头且以“\code{*/}”结尾,或以“\code{(*}”开头且以“\code{*)}”结尾。
	
\end{itemize}

除了以上语法标记外,输入语言中还包括一系列关键字。关键字是一种特殊的词法标记,也叫做保留字,用于在输入文件中表示特殊的含义。输入语言中的关键字如下:
\begin{center}\small
	\begin{tabular}{cccccccc}
		\code{Model} & \code{Var} & \code{Define} & \code{Init} & \code{Transition} & \code{Atomic} & \code{Fairness} & \code{Spec}\\
		\code{int} & \code{bool} & \code{list} & \code{array} & \code{true} & \code{false} & \code{TRUE} & \code{FALSE}\\
		\code{not} & \code{AX} & \code{EX} & \code{AF} & \code{AF} & \code{EG} & \code{AR} & \code{EU}\\
		\code{datatype} & \code{value} & \code{function} & \code{let} & \code{match} & \code{with} & \code{if} & \code{then}\\
		\code{else} &\code{(} &\code{)} &\code{[} & \code{]}& \code{\{} & \code{\}} & \code{=}\\
		\code{!=} &\code{<}& \code{<=}& \code{>} &\code{>=}& \code{+}& \code{+.}& \code{-}\\ 
		\code{-.}& \code{*}& \code{*.} &\code{/}& \code{/.}& \code{!}& \code{|}& \code{||} \\
		\code{\&\&}& \code{->} &\code{:}& \code{::}& \code{,} &\code{;} &\code{.} 
	\end{tabular}
\end{center}
在输入文件中,关键字不能用来定义值或者函数的名字。关键字的用法在接下来的语法描述中有具体描述。

\section{值}
在输入语言中,值通常被用来定义Kripke结构的状态。一个值通常具有以下几种形式:
\begin{itemize}
	\item 单位值:单位值\code{()}通常可看作一个空元组,是最简单的值。
	\item 布尔值:布尔值有两种:\code{true}和\code{false}。
	\item 数值:数值分为两种:整数值与浮点数值。数值的取值范围与OCaml编译器所能表示的相应值的取值范围一致,即:整数值的取值范围为$-2^{30}$到$2^{30}-1$(32位)或$-2^{62}$到$2^{62}-1$(64位);浮点数值的取值范围符合IEEE 754\footnote{\url{https://en.wikipedia.org/wiki/IEEE_754}}标准。
	\item 标量值:标量值也被称为枚举值,标量值通常由编程人员自定义且具有\code{\#iden}形式,即“\code{\#}”符号紧接着一个以小写字母为开头的标识符。
	\item 数组和列表:与OCaml语言类似,数组在这里具有\code{[|v1,...,vn|]}形式,其中\code{v1},...,\code{vn}为$n\ge 0$个值;列表在这里具有\code{[v1,...,vn]}形式,其中\code{v1},...,\code{vn}为$n\ge 0$个值。
	\item 元组:元组具有\code{v1,...,vn},其中\code{v1},...,\code{vn}为$n\ge 1$个值。
	\item 记录:记录可看作带标签的元组,记录具有\code{\{l1 = v1 ; ... ; ln = vn ;\}}形式,其中值\code{vi}的标签是\code{li}。
	\item 变体:变体由变体构造子及若干个值构造而成。变体分两种:只包含构造子的变体(构造子的元数为0)以及构造子和元组构造而成的变体(构造子的元数为元组中值的个数)。
\end{itemize}
值的语法描述如下:
\begin{center}
	\begin{verbatim}
	value ::= 
	     ()
	   | true | false
	   | integer | float
	   | #iden
	   | [| value ; ... ; value |] | [ value ; ... ; value ]
	   | ( value,...,value )
	   | { iden = value ; ... ; iden = value ; }
	   | uiden | uiden ( value , ... , value )
	\end{verbatim}
\end{center}
\section{表达式与模式}
在\sctlprov{}的输入语言中,我们通过表达式的形式来定义复杂的值,即表达式通过有限步规约得到值。为了使表达式更容易理解,我们令表达式的规约顺序与OCaml语言中表达式的规约顺序一致。另外,在模式匹配表达式中,模式的作用是为了匹配不同形式的表达式,其中所定义的模式的匹配形式也与OCaml中一致。
\couic{Expressions are used to build complex values via evluations. The evaluations of expressions in our input language coincide with that of the corresponding OCaml expressions. Values can then be seen as the normal form of expressions, i.e., expressions that cannot be further evaluated.

Patterns are used to match different cases of expressions. The way that patterns match expressions in our input language coincide with that in OCaml.}
\begin{center}
	\begin{verbatim}
	expr ::=
	      value                        (*value*)
	    | iden                         (*variable or name of a value*)
	    | uiden [(expr,...,expr)]      (*variant expression*)
	    | expr . iden                  (*select one field of a record*)
	    | expr [ expr ]                (*select one field of a array*)
	    | ! expr                       (*logical negation*)
	    | expr && expr                 (*logical and*)
	    | expr || expr                 (*logical or*)
	    | - expr                       (*integer negation*)
	    | expr + expr                 (*integer addition*)
	    | expr - expr                 (*integer subtraction*)
	    | expr * expr                 (*integer multiplication*)
	    | -. expr                     (*float negation*)
	    | expr +. expr                (*float addition*)
	    | expr -. expr                (*float subtraction*)
	    | expr *. expr                (*float multiplication*)
	    | expr = expr                 (*expression equivalence*)
	    | expr != expr                (*expression non-equivalence*)
	    | expr < expr                 (*less than*)
	    | expr <= expr                (*less than or equal*)
	    | expr > expr                 (*larger than*)
	    | expr >= expr                (*larger than or equal*)
	    | ( expr )                    (*expression grouping*)
	    | let pattern = expr          (*declare local variables*)
	    | if expr then expr           (*if-then expression*)
	    | if expr then expr else expr (*if-then-else expression *)
	    | expr ; expr                 (*sequence of expressions*)
	    | expr <- expr                (*assignment*)
	    | match_expr                  (*pattern matching*)
	    | expr with {iden = expr ; ... ; iden = expr [;]}  
	                           (*a record with changed bindings*)
	    | iden (expr , ..., expr)     (*a function call*)
	
	match_expr ::= match expr with {| pattern -> expr}+
	
	pattern ::= 
	      iden 
	    | constant
	    | pattern :: pattern                    (*list*)
	    | ( pattern ,..., pattern )             (*tuple*)
	    | _                                     (*match any case*)
	\end{verbatim}
\end{center}

\section{类型}
与传统的编程语言类似,\sctlprov{}的输入语言同样包含类型的定义。类型分为基本类型与复合类型。
\paragraph{基本类型}
\begin{itemize}
	\item 单元类型:单元类型是最简单的类型,记作\code{unit}。这种类型的值只有一个,即\code{()}。
	\item 布尔类型:布尔类型记作\code{bool},该类型的值有两个:\code{true}和 \code{false}。
	\item 整数类型:整数类型可记作\code{int}或者\code{(min .. max)},当记作\code{int}时,该整数类型的取值范围与OCaml的整数取值范围一致;当记作\code{(min .. max)}时,在不超过OCaml的整数类型取值范围的基础上,该整数类型的取值范围为$[min,max]$,其中$min$和$max$均为整数。
	\item 浮点数类型:浮点数类型记作\code{float},与OCaml的浮点数类型的取值范围一致。
	\item 标量类型:标量类型用来表示对有穷个符号的枚举,记作\code{\#iden$_1$, ... , \#iden$_n$},其中\code{iden$_1$, ... , iden$_n$}为任意$n$个符号,即以小写字母开头的标识符。
\end{itemize}
\paragraph{复合类型}
\begin{itemize}
	\item 数组类型:数组类型记作\code{array t},其中\code{t}是一个类型。
	\item 列表类型:列表类型记作\code{list t},其中\code{t}是一个类型。
	\item 元组类型:远足类型记作\code{(t$_1$, ... , t$_n$)},其中\code{t$_1$, ... , t$_n$}为$n>1$个类型。
	\item 记录类型:记录类型可看作带标签的元组类型,记作\code{iden$_1$: t$_1$; ... ; iden$_n$: t$_n$;},其中\code{t$_1$, ... , t$_n$}为$n\ge 1$个类型,而\code{iden$_1$, ... , iden$_n$}为与其对应的$n\ge 1$个标签。
	\item 变体类型:变体类型由若干个变体构造子组成,记作\code{constr$_1$ | ... | constr$_n$},其中\code{constr$_1$ , ... , constr$_n$}为$n\ge 1$个变体构造子,每个构造子具有两种形式:\code{uiden}和\code{uiden t},其中\code{uiden}为以大写字母开头的标识符,而\code{t}为一个类型。
	\item 函数类型:函数类型记作\code{t$_1$ -> t$_2$},其中\code{t$_1$}为参数类型,而\code{t$_2$}为返回值类型。
\end{itemize}
类型的语法定义如下:
\begin{verbatim}
type ::= 
    unit | int | float | bool          (*base types*)
  | (min .. max)                       (*integer type with a range*)
  | {#iden, #iden...,#iden}            (*scalar type*)
  | (type1 , ... , typen)              (*tuple type*)
  | array type                         (*array type*)
  | list type                          (*list type*)
  | record_type                        (*record type*)
  | variant_type                       (*variant type*)
  | type -> type                       (*function type*)

variant_type ::= constr | ... | constr
constr  ::= uiden | uiden type
record_type  ::= { iden : type ; ... ; iden : type ;}
\end{verbatim}
\section{值、类型以及函数的声明}
在\sctlprov{}的输入文件中,为了程序的可读性以及方便书写,我们可用标识符来代指先前所定义的值与类型以及声明函数。

\begin{itemize}
	\item 值与类型的声明:值的声明语法如下。
	\begin{verbatim}
	value_def ::= value iden = expr		(*value definition*)
	\end{verbatim}
	类型的声明语法如下。
	\begin{verbatim}
	type_decl ::=  datatype iden = type (*to define new types*)
	\end{verbatim}
	\item 函数的声明:函数的声明语法如下。
	\begin{verbatim}
	fun_decl   ::=  
	     function iden ( parameter,...,parameter ) : type = expr
	parameter  ::=  iden | ( parameter,...,parameter )
	\end{verbatim}
\end{itemize}

\section{Kripke模型的声明}
以上介绍的所有语法结构的目的是为了定义Kripke模型。在\sctlprov{}的输入语言中,一个Kripke模型的定义分为6个部分:状态变量声明、初始状态、迁移规则、原子公式、公平性约束以及性质描述。
\begin{enumerate}
	\item 状态变量声明:在传统的模型检测工具的输入语言中,状态被定义为对若干个状态变量的赋值,对状态变量的不同赋值代表了不同的状态。在\sctlprov{}的输入语言中同样保留了这种定义状态的方式。然而,一个更一般化的定义状态的方式是将状态定义为值,即用一个值来表示一个状态。这样定义的状态就可以具有更复杂的类型,从而在表示复杂系统的建模中更方便。因此,状态变量声明在输入文件中被作为一个可选项。
	\item 初始状态:当输入文件中存在状态变量声明,那么Kripke模型中必须有初始状态的定义。此时,初始状态被定义为对所声明的所有状态变量的一个赋值。当输入文件中不存在状态变量的声明时,初始状态就可用值的声明来定义。
	\item 迁移规则:迁移规则的定义分为两种,第一种类似于传统的模型检测工具的输入语言中的定义,即列出若干个布尔表达式及其对应的状态变换方式,依次规约布尔表达式,并且当布尔表达式规约到值\code{true}时依据其对应的状态变换方式进行状态的迁移;第二种定义迁移规则的方式则更一般化:将要迁移的状态作为参数传递给一个函数,该函数会返回一个状态的列表,那么该列表中的所有状态则为当前要迁移的状态的后继。
	\item 原子公式:原子公式被定义为布尔表达式,判断原子公式的满足性时,需将若干个状态作为参数带入到相应的布尔表达式中,如果代入后的布尔表达式可规约到值\code{true}则该原子公式为真,否则为假。
	\item 公平性约束:公平性约束指的是若干个\CTLP{}公式的定义。
	\item 性质描述:Kripke模型的性质同样被定义为若干个\CTLP{}公式,不同于公平性约束,在性质描述种,为了可读性与叙述验证结果方便,通常将每个性质绑定到一个标识符上。
\end{enumerate}
Kripke模型声明的语法如下。
\begin{verbatim}
kripke_decl ::=  
   Model { 
     (*Define states as lists of state variable bindings (a record), 
     this is optional.*)
     Vars { iden : type ; ... ; iden : type ; }
     (*Define the initial state (refered to as either "ini" or 
     "init"), also optional.*)
     Init { iden := expr ; ... ; iden := expr ; }
     (*Define the transition relation.*)
     Transition { next iden := transitions }
     (*Define atomic formulae.*)
     Atomic { {iden ( iden , ... , iden ) := expr ;}* }		
     (*Define fairness constraints, optional.*)
     Fairness { formula ; ... ; formula }
     (*Define the specification.*)
     Spec { iden := formula ; ... ; iden := formula ;}
   }
trasitions ::= expr | expr : expr ; ... ; expr : expr ;
formula ::= 
     iden ( iden , ... , iden )
   | not formula
   | formula /\ formula
   | formula \/ formula
   | EX ( iden , formula , expr )
   | AX ( iden , formula , expr )
   | EG ( iden , formula , expr )
   | AF ( iden , formula , expr )
   | EU ( iden , iden , formula , formula , expr )
   | AR ( iden , iden , formula , formula , expr )
\end{verbatim}
\section{输入文件的结构}
\sctlprov{}在每次验证过程中可接受一个或多个输入文件。当只有一个输入文件时,那么该输入文件被作为主文件输入到\sctlprov{}中。主文件中包含若干个值、类型以及函数的声明,以及一个Kripke模型的声明。当有多个输入文件中,除了一个主文件以外,还有若干个模块文件,模块文件中只可包含值、类型或者函数的定义,而不包含Kripke模型的定义。多个输入文件利用\code{import}语句来确定依赖关系,即一个模块文件的内容可通过\code{import}引入到别的输入文件中。\code{import}语句的参数为以大写字母开头的标识符,这个标识符代表的即为要引入的模块文件的文件名(如果文件名以小写字母开头,则将其起始字母变为大写字母)。
\begin{verbatim}
module ::= 
    import uiden ... import uiden
    decl ... decl
decl   ::= type_decl | value_decl | fun_decl
main_file ::= kripke_decl | module kripke_decl
\end{verbatim}
\couic{



\section{输入文件的结构}
Programs are organized as modules. Each modules contains a set of declarations. Modules can be imported into others, while cyclic dependencies are not allowed.


}
\cleardoublepage

%\chapter{\textsf{SATS}模型的输入文件}\label{appendix:sats}
\begin{verbatim}
datatype side = Right | Left

datatype aircraft = {
seq: int;
mahf: side;
}

datatype sca = {
holding3_left: 	list aircraft;
holding3_right: list aircraft;
holding2_left: 	list aircraft;
holding2_right: list aircraft;
lez_left: 		list aircraft;
lez_right:		list aircraft;
maz_left:		list aircraft;
maz_right:		list aircraft;
base_left:		list aircraft;
base_right:		list aircraft;
departure_left:	list aircraft;
departure_right:list aircraft;
intermediate:	list aircraft;
final:			list aircraft;
runway:			list aircraft;
}

// helper functions for list of aircrafts
function first_of_list (la) : (list aircraft) -> aircraft = 
match la with
| [a] -> a
| a :: as -> a

function remove_first (la) : (list aircraft) -> (list aircraft) = 
match la with
| [a] -> []
| a :: as -> as

function add_to_last (la, a) : (list aircraft, aircraft) -> (list aircraft) =
match la with
| [] -> [a]
| a1 :: as -> a1 :: (add_to_last (as, a))

function length (la) : (list aircraft) -> int = 
match la with
| [] -> 0
| a :: as -> 1 + length (as)

function opposite (s) : side -> side = 
match s with
| Left -> Right
| Right -> Left

value none = 0
value first = 1

function sign (s): side -> int = 
match s with
| Left -> 1
| Right -> -1

// Is this the first aircraft in landing sequence?
function is_first (a) : aircraft -> bool = 
a.seq = first

// Leading sequence of leader aircraft
function leader (a) : aircraft -> int = 
if a.seq = none then none else (a.seq - 1)

// Is b the leader aircraft of a?
function is_leader (a, b) : (aircraft, aircraft) -> bool = 
b.seq = leader (a)

// Next landing sequence
function next_landing (a) : aircraft -> int =
a.seq + 1

// Number of aircraft in a zone (list of aircrafts) to assingned to one side
function assigned (z, s) : ((list aircraft), side) -> int = 
if z = [] then 
0
else 
let (fz, rz) = (first_of_list (z), remove_first (z));
if fz.mahf = s then 
1 + (assigned (rz, s))
else 
assigned (rz, s)


// Is any aircraft in zone z assigned to the mahf side?
function is_assigned (z, s) : (list aircraft, side) -> bool = 
assigned (z, s) != 0

// Is aircraft with sequence number n in zone z?
function is_on_zone (z, n) : (list aircraft, int) -> bool = 
match z with
| [] -> false
| z1::zs -> if z1.seq = n then true else is_on_zone (zs, n)

// Decrease by one the sequence number of aircraft in zone z
function decrease (z): (list aircraft) -> (list aircraft) = 
match z with
| [] -> []
| z1 :: zs -> (z1 with {seq = leader (z1);}) :: (decrease (zs))

// Is aircraft with sequence number n on the approach?
function is_on_approach (s, n) : (sca, int) -> bool = 
is_on_zone (s.base_right, n) || 
is_on_zone (s.base_left, n) || 
is_on_zone (s.intermediate, n) || 
(is_on_zone (s.final, n))

// Is any aircraft on the approach assigned to the mahf side?
function is_on_approach_side (s, si) : (sca, side) -> bool = 
is_assigned (s.base_right, si) || is_assigned (s.base_left, si) || is_assigned (s.intermediate, si) || is_assigned (s.final, si)

// Actual number of aircraft at one side (excluding the approach)
function actual (s, si) : (sca, side) -> int = 
match si with
| Left -> length (s.holding3_left) + length (s.holding2_left) + length (s.lez_left) + length (s.maz_left)
| Right -> length (s.holding3_right) + length (s.holding2_right) + length (s.lez_right) + length (s.maz_right)

// Virtual number of aircraft at one fix
function virtual (s, si) : (sca, side) -> int = 
match si with
| Left -> 
length (s.holding3_left) + length (s.holding2_left) + length (s.lez_left) + length (s.maz_left) +
assigned (s.holding3_right, Left) + assigned (s.holding2_right, Left) + assigned (s.lez_right, Left) +
assigned (s.maz_right, Left) + assigned (s.base_right, Left) + assigned (s.base_left, Left) +
assigned (s.intermediate, Left) + assigned (s.final, Left)
| Right -> 
length (s.holding3_right) + length (s.holding2_right) + length (s.lez_right) + length (s.maz_right) +
assigned (s.holding3_left, Right) + assigned (s.holding2_left, Right) + assigned (s.lez_left, Right) +
assigned (s.maz_left, Right) + assigned (s.base_right, Right) + assigned (s.base_left, Right) +
assigned (s.intermediate, Right) + assigned (s.final, Right)

// Number of aircraft assigned to a fix
function assigned2fix (s, si) : (sca, side) -> int = 
assigned (s.holding3_left, si) + assigned (s.holding3_right, si) +
assigned (s.holding2_left, si) + assigned (s.holding2_right, si) +
assigned (s.lez_left, si) + assigned (s.lez_right, si) +
assigned (s.base_left, si) + assigned (s.base_right, si) +
assigned (s.intermediate, si) +
assigned (s.final, si) +
assigned (s.maz_left, si) + assigned (s.maz_right, si)

// Total number of simultaneous landing operations
function landing_op (s) : sca -> int = 
actual (s, Left) + actual (s, Right) +
length (s.base_left) + length (s.base_right) +
length (s.intermediate) + length (s.final)


value dummy_aircraft : aircraft = {
mahf = Left;
seq = -1;
}

//find aircraft with seq number x
function find_aircraft_by_seq (ta, x) : (list aircraft, int) -> aircraft = 
match ta with
| [] -> dummy_aircraft
| a :: tas -> 
if a.seq = x then 
a
else
find_aircraft_by_seq (tas, x)

// New aircraft
function new_aircraft (s, si) : (sca, side) -> aircraft = 
let ta = all_in_sca (s);
if length (ta) = 0 then
{
mahf = si;
seq = 1;
}
else
let last = find_aircraft_by_seq (ta, length(ta));
{
mahf = (opposite(last.mahf));
seq = length (ta) + 1;
}

// New aircraft for departure
function departure (s, si) : (sca, side) -> aircraft = {
mahf = si;
seq = 0;
}

// Reassign aircraft
function reassign (s, a) : (sca, aircraft) -> aircraft = 
let all = all_in_sca (s);
let tn = length (all);
let last = find_aircraft_by_seq(all, tn);
a with {
mahf = (if tn = first then a.mahf else opposite(last.mahf));
seq = tn + 1;
}

// Move a departing aircraft
function move (a) : aircraft -> aircraft = 
a with {
seq = (if a.seq = 0 then 3 else 10);
}

// **************************Now define the helper functions for the transition rules***********************************
function vertical_entry (si, s) : (side, sca) -> (list sca) = 
if virtual (s, si) < 2 && 
(!(is_on_approach_side (s, si))) && 
((length (if si = Left then s.maz_left else s.maz_right)) = 0) &&
((length (if si = Left then s.lez_left else s.lez_right)) = 0) && 
((length (if si = Left then s.holding3_left else s.holding3_right)) = 0) then
let a = new_aircraft (s, si);
match si with
| Left -> 
[s with {
holding3_left = add_to_last (s.holding3_left, a);
}]
| Right -> 
[s with {
holding3_right = add_to_last (s.holding3_right, a);
}]
else
[]

function lateral_entry (si, this) : (side, sca) -> (list sca) = 
if virtual (this, si) = 0 then
let a = new_aircraft (this, si);
match si with
| Left -> 
[this with{
lez_left = add_to_last (this.lez_left, a);
}]
| Right -> 
[this with{
lez_right = add_to_last (this.lez_right, a);
}]
else
[]

function holding_pattern_descend (si, this) : (side, sca) -> (list sca) = 
match si with
| Left ->
if (length (this.holding3_left) != 0) && (length (this.holding2_left) = 0) then
(let a = first_of_list (this.holding3_left);
[this with {
holding3_left = remove_first (this.holding3_left);
holding2_left = add_to_last (this.holding2_left, a);
}])
else
[]
| Right ->
if (length (this.holding3_right) != 0) && (length (this.holding2_right) = 0) then
(let a = first_of_list (this.holding3_right);
[this with {
holding3_right = remove_first (this.holding3_right);
holding2_right = add_to_last (this.holding2_right, a);
}])
else
[]

function vertical_approach_initiation (si, this) : (side, sca) -> (list sca) = 
match si with
| Left ->
if length (this.holding2_left) != 0 then
(let a = first_of_list (this.holding2_left);
if length (this.base_right) <= 1 && (is_first (a) || is_on_approach (this, leader (a))) then
[this with {
holding2_left = remove_first (this.holding2_left);
base_left = add_to_last (this.base_left, a);
}]
else
[])
else
[]
| Right ->
if length (this.holding2_right) != 0 then
(let a = first_of_list (this.holding2_right);
if length (this.base_left) <= 1 && (is_first (a) || is_on_approach (this, leader (a))) then
[this with {
holding2_right = remove_first (this.holding2_right);
base_right = add_to_last (this.base_right, a);
}]
else
[])
else
[]

function lateral_approach_initiation (si, this) : (side, sca) -> (list sca) = 
match si with
| Left ->
if length (this.lez_left) != 0 then
(let a = first_of_list (this.lez_left);
if length (this.base_right) <= 1 && (is_first (a) || is_on_approach (this, leader (a))) then
[this with {
lez_left = remove_first (this.lez_left);
base_left = add_to_last (this.base_left, a);
}]
else
[this with {
lez_left = remove_first (this.lez_left);
holding2_left = add_to_last (this.holding2_left, a);
}])
else 
[]
| Right ->
if length (this.lez_right) != 0 then
(let a = first_of_list (this.lez_right);
if length (this.base_left) <= 1 && (is_first (a) || is_on_approach (this, leader (a))) then
[this with {
lez_right = remove_first (this.lez_right);
base_right = add_to_last (this.base_right, a);
}]
else
[this with {
lez_right = remove_first (this.lez_right);
holding2_right = add_to_last (this.holding2_right, a);
}])
else 
[]

function merging (si, this) : (side, sca) -> (list sca) = 
match si with
| Left ->
if length (this.base_left) != 0 then
(let a = first_of_list (this.base_left);
(if is_first (a) || is_on_zone (this.intermediate, leader (a)) || is_on_zone (this.final, leader (a)) then
[this with {
base_left = remove_first (this.base_left);
intermediate = add_to_last (this.intermediate, a);
}]
else
[]))
else
[]
| Right ->
if length (this.base_right) != 0 then
(let a = first_of_list (this.base_right);
(if is_first (a) || is_on_zone (this.intermediate, leader (a)) || is_on_zone (this.final, leader (a)) then
[this with {
base_right = remove_first (this.base_right);
intermediate = add_to_last (this.intermediate, a);
}]
else
[]))
else 
[]

function exit this : sca -> (list sca) = 
if length (this.intermediate) != 0 && is_first (first_of_list (this.intermediate)) then //assume "first" is in PVS
[this with {
holding3_right = decrease (this.holding3_right);
holding3_left = decrease (this.holding3_left);
holding2_right = decrease (this.holding2_right);
holding2_left = decrease (this.holding2_left);
lez_right = decrease (this.lez_right);
lez_left = decrease (this.lez_left);
maz_right = decrease (this.maz_right);
maz_left = decrease (this.maz_left);
base_right = decrease (this.base_right);
base_left = decrease (this.base_left);
intermediate = decrease (remove_first (this.intermediate));
final = decrease (this.final);
}]
else
[]

function final_segment this : sca -> (list sca) = 
if length (this.intermediate) != 0 then
(let a = first_of_list (this.intermediate);
[this with {
intermediate = remove_first (this.intermediate);
final = add_to_last (this.final, a);
}])
else 
[]

function landing this : sca -> (list sca) = 
if length (this.final) != 0 && length (this.runway) = 0 then
[this with {
holding3_right = decrease (this.holding3_right);
holding3_left = decrease (this.holding3_left);
holding2_right = decrease (this.holding2_right);
holding2_left = decrease (this.holding2_left);
lez_right = decrease (this.lez_right);
lez_left = decrease (this.lez_left);
maz_right = decrease (this.maz_right);
maz_left = decrease (this.maz_left);
base_right = decrease (this.base_right);
base_left = decrease (this.base_left);
intermediate = decrease (this.intermediate);
final = decrease (remove_first (this.final));
runway = add_to_last (this.runway, (first_of_list (this.final)));
}]
else
[]

function taxing this : sca -> (list sca) =
if length (this.runway) != 0 && is_first (first_of_list (this.runway)) then
[this with {
runway = remove_first (this.runway);
}]
else
[]

function missed_approach this : sca -> (list sca) =
if (length (this.final) != 0) then
(let a = first_of_list (this.final);
let ra = reassign (this, a);
match a.mahf with
| Left ->
[this with {
holding3_right = decrease (this.holding3_right);
holding3_left = decrease (this.holding3_left);
holding2_right = decrease (this.holding2_right);
holding2_left = decrease (this.holding2_left);
lez_right = decrease (this.lez_right);
lez_left = decrease (this.lez_left);
maz_right = decrease (this.maz_right);
maz_left = decrease (add_to_last (this.maz_left, ra));
base_right = decrease (this.base_right);
base_left = decrease (this.base_left);
intermediate = decrease (this.intermediate);
final = decrease (remove_first (this.final)); 
}]
| Right ->
[this with {
holding3_right = decrease (this.holding3_right);
holding3_left = decrease (this.holding3_left);
holding2_right = decrease (this.holding2_right);
holding2_left = decrease (this.holding2_left);
lez_right = decrease (this.lez_right);
lez_left = decrease (this.lez_left);
maz_right = decrease (add_to_last (this.maz_right, ra));
maz_left = decrease (this.maz_left);
base_right = decrease (this.base_right);
base_left = decrease (this.base_left);
intermediate = decrease (this.intermediate);
final = decrease (remove_first (this.final));
}])
else 
[]

function lowest_avaliable_altitude (si, this) : (side, sca) -> (list sca) =
match si with
| Left ->
if length (this.maz_left) != 0 then
let a = first_of_list (this.maz_left);
(if length (this.holding3_left) = 0 && length (this.holding2_left) = 0 then
[this with {
holding2_left = add_to_last (this.holding2_left, a);
maz_left = remove_first (this.maz_left);
}]
else (if length (this.holding3_left) = 0 then
[this with {
holding3_left = add_to_last (this.holding3_left, a);
maz_left = remove_first (this.maz_left);
}]
else
[this with {
holding3_left = add_to_last ((remove_first (this.holding3_left)), a);
holding2_left = add_to_last (this.holding2_left, (first_of_list (this.holding3_left)));
maz_left = remove_first (this.maz_left);
}]))
else
[]
| Right ->
if length (this.maz_right) != 0 then
let a = first_of_list (this.maz_right);
(if length (this.holding3_right) = 0 && length (this.holding2_right) = 0 then
[this with {
holding2_right = add_to_last (this.holding2_right, a);
maz_right = remove_first (this.maz_right);
}]
else (if length (this.holding3_right) = 0 then
[this with {
holding3_right = add_to_last (this.holding3_right, a);
maz_right = remove_first (this.maz_right);
}]
else
[this with {
holding3_right = add_to_last ((remove_first (this.holding3_right)), a);
holding2_right = add_to_last (this.holding2_right, (first_of_list (this.holding3_right)));
maz_right = remove_first (this.maz_right);
}]))
else
[]

function seq_of_aircraft (a) : aircraft -> int = 
a.seq			

function departure_initiation (si, this) : (side, sca) -> (list sca) = 
match si with
| Left ->
if (length (this.final) + length (this.runway)) = 0 &&
(length (this.departure_right) + length (this.departure_left) < 2) &&
((length (this.departure_right) + length (this.departure_left)) = 0 ||
(length (this.departure_right) > 0 && (seq_of_aircraft (first_of_list (this.departure_right))) >= 3) ||
(length (this.departure_left) > 0 && (seq_of_aircraft (first_of_list (this.departure_left))) >= 10)
) then
[this with {
runway = add_to_last (this.runway, departure (this, Left));
}]
else 
[]
| Right ->
if (length (this.final) + length (this.runway)) = 0 &&
(length (this.departure_right) + length (this.departure_left) < 2) &&
((length (this.departure_right) + length (this.departure_left)) = 0 ||
(length (this.departure_left) > 0 && (seq_of_aircraft (first_of_list (this.departure_left))) >= 3) ||
(length (this.departure_right) > 0 && (seq_of_aircraft (first_of_list (this.departure_right))) >= 10)
) then
[this with {
runway = add_to_last (this.runway, departure (this, Right));
}]
else 
[]

function takeoff (this) : sca -> (list sca) = 
if length (this.runway) > 0 then
let a = first_of_list (this.runway);
if a.seq = 0 then
match a.mahf with
| Left ->
[this with {
runway = remove_first (this.runway);
departure_left = add_to_last (this.departure_left, a);
}]
| Right ->
[this with {
runway = remove_first (this.runway);
departure_right = add_to_last (this.departure_right, a);
}]
else 
[]
else 
[]

function departing (si, this) : (side, sca) -> (list sca) =
match si with
| Left ->
if length (this.departure_left) > 0 then
(let a = first_of_list (this.departure_left);
let new_dep_left = 
(if a.seq >= 10 then
remove_first (this.departure_left)
else
((move (a)) :: (remove_first (this.departure_left))));
[this with {
departure_left = new_dep_left;
}])
else 
[]
| Right ->
if length (this.departure_right) > 0 then
(let a = first_of_list (this.departure_right);
let new_dep_right = 
(if a.seq >= 10 then
remove_first (this.departure_right)
else
((move (a)) :: (remove_first (this.departure_right))));
[this with {
departure_right = new_dep_right;
}])
else 
[]

// **********************Defining related properties**************************
// No more than four simultaneous landing operations
function four_landings (s) : sca -> bool = 
landing_op (s) <= 4 && ((length (all_in_sca (s))) <= landing_op (s))

// No more than 2 aircraft assigned to each missed approach fix. Moreover,
// no more than 2 aircraft at each side of the sca (excluding the approach).
function well_assigned (s) : sca -> bool = 
assigned2fix (s, Right) <= 2 && assigned2fix (s, Left) <= 2 &&
actual (s, Right) <= 2 && actual (s, Left) <= 2

// At most one aircraft at a holding fix for a given altitude.
// No more than 2 aircraft on a missed approach zone.
function non_crowded_side (si, s) : (side, sca) -> bool = 
match si with
| Left ->
length (s.holding3_left) <= 1 && length (s.holding2_left) <= 1 && length (s.maz_left) <= 2
| Right ->
length (s.holding3_right) <= 1 && length (s.holding2_right) <= 1 && length (s.maz_right) <= 2

// Each side of the sca is non-crowded. Furthermore, no more than 3 aircraft on base.
function non_crowded_sca (s) : sca -> bool = 
non_crowded_side (Left, s) && non_crowded_side (Right, s) && (length (s.base_right) + length (s.base_left) <= 3)

// At most one aircraft on lateral entry. Moreover, if there is an aircraft on lateral entry, 
// that side of the sca is empty (excluding the approach)
function safe_len (si, s) : (side, sca) -> bool = 
match si with
| Left ->
length (s.lez_left) <= 1 && 
((length (s.lez_left) <= 0) || (length (s.holding3_left) + length (s.holding2_left) + length (s.maz_left) = 0))
| Right ->
length (s.lez_right) <= 1 && 
((length (s.lez_right) <= 0) || (length (s.holding3_right) + length (s.holding2_right) + length (s.maz_right) = 0))

// An aircraft on the approach is either the first in sequence or its leader is already in the approach
function smooth_merging (si, s) : (side, sca) -> bool = 
match si with
| Left ->
if length (s.base_left) = 0 then 
true
else
(let a = first_of_list (s.base_left);
(is_first (a) || 
(is_on_zone (s.intermediate, leader (a))) || 
(is_on_zone (s.final,leader (a))) || 
((length (s.base_right)) > 0 && 
(is_leader (a, first_of_list (s.base_right))))))
| Right ->
if length (s.base_right) = 0 then
true
else
(let b = first_of_list (s.base_right);
((is_first (b)) ||
(is_on_zone (s.intermediate, leader (b))) ||
(is_on_zone (s.final, leader (b))) ||
(length (s.base_left) > 0 && (is_leader (b, first_of_list (s.base_left))))))

// Aircraft land in sequence order
function safe_landing (s) : sca -> bool = 
length (s.final) <= 0 || (is_first (first_of_list (s.final)))

// At most one aircraft at the runway
function no_incursion (s) : sca -> bool = 
length (s.runway) <= 1

// Simultaneous departure are separated (3nm if using different departure fix, 10 nm if using the same departure fix).
function safe_departure (si, s) : (side, sca) -> bool = 
match si with
| Left ->
if length (s.departure_left) <= 1 then
true 
else 
let (fst_dep_left, fst_rest_dep_left) = (first_of_list (s.departure_left), first_of_list (remove_first (s.departure_left)));
fst_dep_left.seq = 10 && fst_rest_dep_left.seq = 0
| Right ->
if length (s.departure_right) <= 1 then
true 
else 
let (fst_dep_right, fst_rest_dep_right) = (first_of_list (s.departure_right), first_of_list (remove_first (s.departure_right)));
fst_dep_right.seq = 10 && fst_rest_dep_right.seq = 0

// All the above properties
function invariant (s) : sca -> bool = 
four_landings (s) &&
well_assigned (s) &&
non_crowded_sca (s) &&
safe_len (Right, s) &&
safe_len (Left, s) &&
smooth_merging (Right, s) &&
smooth_merging (Left, s) &&
safe_landing (s) &&
no_incursion (s) &&
safe_departure (Right, s) &&
safe_departure (Left, s)

// Initial state
value init : sca = {
departure_left = [];
departure_right = [];
runway = [];
final = [];
intermediate = [];
base_left = [];
base_right = [];
maz_left = [];
lez_left = [];
holding2_left = [];
holding3_left = [];
maz_right = [];
lez_right = [];
holding2_right = [];
holding3_right = [];
}

//all aircraft in the sca
function all_in_sca (s) : sca -> (list aircraft) = 
s.holding3_left @ s.holding3_right
@ s.holding2_left @ s.holding2_right
@ s.lez_left @ s.lez_right
@ s.maz_left @ s.maz_right
@ s.base_left @ s.base_right
@ s.intermediate
@ s.final


// SCA will be empty again
function empty_again this : sca -> bool = 
length (this.holding3_left) = 0 && length (this.holding3_right) = 0 &&
length (this.holding2_left) = 0 && length (this.holding2_right) = 0 &&
length (this.lez_left) = 0 && length (this.lez_right) = 0 &&
length (this.maz_left) = 0 && length (this.maz_right) = 0 &&
length (this.departure_left) = 0 && length (this.departure_right) = 0 &&
length (this.intermediate) = 0 && length (this.final) = 0 &&
length (this.runway) = 0

Model SATS () 
{
Transition {
next s := 
(vertical_entry (Left, s)) @ (vertical_entry (Right, s)) @
(lateral_entry (Left, s)) @ (lateral_entry (Right, s)) @
(holding_pattern_descend (Left, s)) @ (holding_pattern_descend (Right, s)) @
(vertical_approach_initiation (Left, s)) @ (vertical_approach_initiation (Right, s)) @
(lateral_approach_initiation (Left, s)) @ (lateral_approach_initiation (Right, s)) @
(merging (Left, s)) @ (merging (Right, s)) @
(exit (s)) @
(final_segment (s)) @
(landing (s)) @
(taxing (s)) @
(missed_approach (s)) @
(lowest_avaliable_altitude (Left, s)) @ (lowest_avaliable_altitude (Right, s)) @ 
(departure_initiation (Left, s)) @ (departure_initiation (Right, s)) @
(takeoff (s)) @
(departing (Left, s)) @ (departing (Right, s))
}

Atomic {
atom_safe (s) := invariant (s);
atom_empty_again (s) := empty_again (s);
}

Spec {
safe := AR (x, y, FALSE, atom_safe (y), init);
emergency := AX (x, AF(y, atom_empty_again (y), x), init);
}
}

\end{verbatim}

\chapter{\tool{VMDV}与定理证明工具的交互协议}\label{vmdv:json:protocol}
\tool{VMDV}与不同的定理证明工具的交互是通过互相发送与解析JSON形式的消息来实现的,每个JSON对象代表一个消息,消息的详细说明如下:
\begin{itemize}
	\item \textbf{初始化显示窗口}({定理证明器} $\longrightarrow$ \tool{VMDV}):在\tool{VMDV}中,我们将每个显示窗口成为一个会话(Session)。当定理证明器需要可视化一个数据结构(比如证明树)的时候,定理证明器发送一个“create\_session”消息, 指定需要可视化的数据结构的名称(“session\_id”)、简介(“session\_descr”)及类型(“graph\_type”)。其中,定理证明器指定的可视化的数据结构的类型分为两种:树(“Tree”)和有向图(“DiGraph”),前者通常用来表示证明树,后者通常用来表示可能有环的有向图。\tool{VMDV}接收到该消息之后立即初始化并显示一个窗口,并等待定理证明器的进一步消息。
	\begin{verbatim}
	{
	    "type": "create_session",
	    "session_id": string,
	    "session_descr": string,
	    "graph_type": string
	}
	\end{verbatim}
	\item \textbf{添加节点}(定理证明器 $\longrightarrow$ \tool{VMDV}):向\tool{VMDV}相应的会话中加入一个节点,并指定节点的标识(“id”)、要显示的字符串(“label”)、状态(“state”)。其中,节点的状态分为两种:已证明(“Proved”)和未证明(“Not\_Proved”),不同状态的节点通常用不同的颜色高亮显示。\couic{已证明的节点通常默认高亮为绿色,而未证明的节点通常默认高亮为黄色。}
	\begin{verbatim}
	{
	    "type": "add_node",
	    "session_id": string,
	    "node":
	    {
	        "id": string,
	        "label": string,
	        "state": string
	    }
	}
	\end{verbatim}
	\item \textbf{删除节点}(定理证明器 $\longrightarrow$ \tool{VMDV}):在指定的会话中删除一个指定的节点。
	\begin{verbatim}
	{
	    "type": "remove_node",
	    "session_id": string,
	    "node_id": string
	}
	\end{verbatim}
	\item \textbf{添加边}(定理证明器 $\longrightarrow$ \tool{VMDV}):在指定的会话中添加一条由“from\_id”节点指向“to\_id”节点的边,同时指定该条边上要显示的字符串(“label”)。
	\begin{verbatim}
	{
	    "type": "add_edge",
	    "session_id": string,
	    "from_id": string,
	    "to_id": string,
	    "label": string
	}
	\end{verbatim}
	\item \textbf{删除边}(定理证明器 $\longrightarrow$ \tool{VMDV}):在指定的会话中删除由“from\_id”节点指向“to\_id”节点的边。
	\begin{verbatim}
	{
	    "type": "remove_node",
	    "session_id": string,
	    "node_id": string
	}
	\end{verbatim}
	\item \textbf{高亮节点}(定理证明器 $\longleftrightarrow$ \tool{VMDV}):高亮节点消息可由定理证明器和\tool{VMDV}互相发送给对方。当定理证明器向\tool{VMDV}发送该消息时,\tool{VMDV}在相应的会话中将指定的节点用不同于节点当前的颜色高亮显示。当用户在\tool{VMDV}窗口中执行鼠标选中或者搜索操作时,被选中或者搜索到的节点在窗口中被高亮显示的同时,将所有节点的高亮消息发送到定理证明器。当定理证明器同时在\tool{VMDV}中可视化多个数据结构时,高亮节点消息的传递过程可用来实现多个数据结构的交互显示。比如,在可视化\sctlprov{}的证明结果的时候,\tool{VMDV}通常会初始化两个窗口,分别可视化当前公式的证明树,以及在搜索当前证明树的过程中所访问到的Kripke模型的状态。当用鼠标选中或搜索到证明树的某个或某些节点时,\tool{VMDV}将这些证明树节点的高亮信息发送给\sctlprov{},同时\sctlprov{}识别相应的证明树节点,并计算出与这些证明树节点相关的状态,然后将这些状态节点的高亮信息发送给\tool{VMDV},于是\tool{VMDV}在状态图窗口中高亮显示相应的状态。
	
	%	在指定的会话中删除由“from\_id”节点指向“to\_id”节点的边。
	\begin{verbatim}
	{
	    "type": "remove_node",
	    "session_id": string,
	    "node_id": string
	}
	\end{verbatim}
	\item \textbf{取消高亮节点}(定理证明器 $\longleftrightarrow$ \tool{VMDV}):同高亮节点消息一样,取消高亮节点的消息同样可由定理证明器和\tool{VMDV}互相发送给对方。
	\begin{verbatim}
	{
	    "type": "unhighlight_node",
	    "session_id": string,
	    "node_id": string
	}
	\end{verbatim}
	\item \textbf{取消所有高亮}(定理证明器 $\longleftrightarrow$ \tool{VMDV}):去掉所有节点的高亮颜色。
	\begin{verbatim}
	{
	    "type": "clear_color",
	    "session_id": string
	}
	\end{verbatim}
	\item \textbf{删除会话} (定理证明器 $\longrightarrow$ \tool{VMDV}):\tool{VMDV}收到删除会话消息后,销毁对应的窗口,结束一个数据结构的可视化过程。
	\begin{verbatim}
	{
	    "type": "remove_session",
	    "session_id": string
	}
	\end{verbatim}
\end{itemize}

当定理证明器或\tool{VMDV}收到对方发来的控制消息后,会发送回对方一个反馈消息,用来表明该控制消息是否被成功解析:
\begin{itemize}
	\item \textsf{成功解析}(定理证明器 $\longleftrightarrow$ \tool{VMDV}):
	\begin{verbatim}
	{
	    "type": "feedback",
	    "session_id": string,
	    "status": "OK"
	}
	\end{verbatim}
	\item \textbf{解析失败}(定理证明器 $\longleftrightarrow$ \tool{VMDV}):
	\begin{verbatim}
	{
	    "type": "feedback",
	    "session_id": string,
	    "status": "Fail",
	    "error_msg": string
	}
	\end{verbatim}
\end{itemize}
\cleardoublepage
\chapter{部分实验结果的详细数据}\label{appendix:detail:data}

\begin{figure}[h!]\scriptsize
	\centering
	%	\begin{tabular}{p{6.8cm} p{6.8cm}}
	%	\begin{figure}
	%		\setlength{\tabcolsep}{2pt}
	%		\renewcommand{\arraystretch}{1.0}
	%		\begin{minipage}[b]{0.5\linewidth}
	\setlength{\tabcolsep}{8pt}
	\renewcommand\arraystretch{0.8}
	\begin{tabular}{| r | r | r | r | r | r | r | r | r | r |}
		\hline
		\textbf{性质} & \textbf{进程数} & \multicolumn{8}{c|}{互斥算法} \\
		\hline
		{} & {} & \multicolumn{2}{c|}{\verds{}} & \multicolumn{2}{c|}{\nusmv{}} & \multicolumn{2}{c|}{\nuxmv{}} &  \multicolumn{2}{c|}{\sctl{}} \\
		\hline
		{} & {} & sec & MB & sec & MB & sec & MB &  sec & MB \\
		\hline
		\multirow{6}{*}{$P_1$} & 6 & 0.286 & 321.99 & 0.153 & 9.07 & 0.270 & 21.18 & 0.005 & 2.25 \\
		%			\hline
		{} & 12 & 1.278 & 322.08 & 19.506 & 76.98 & 21.848 & 89.25 & 0.016 & 3.70  \\
		%			\hline
		{} & 18 & 4.719 & 426.45 & - & - & - & - & 0.037 & 5.44  \\
		%			\hline
		{} & 24 & 11.989 & 601.55 & - & - & - & - & 0.091 & 9.36  \\
		%			\hline
		{} & 30 & 26.511 & 926.25 & - & - & - & - & 0.200 & 16.49  \\
		%			\hline
		{} & 36 & 52.473 & 1287.57 & - & - & - & - & 0.418 & 27.46  \\
		%			\hline
		{} & 42 & 100.071 & 1944.95 & - & - & - & - & 0.682 & 48.28  \\
		%			\hline
		{} & 48 & - & - & - & - & - & - & 1.119 & 66.63  \\
		%			\hline
		{} & 51 & - & - & - & - & - & - & 1.392 & 82.32  \\
		\hline
		\multirow{6}{*}{$P_2$} & 6 & 0.375 & 322.07 & 0.054 & 9.07 & 0.048 & 21.31 & 0.012 & 3.07  \\
		%			\hline
		{} & 12 & 2.011 & 322.02 & 22.774 & 76.96 & 21.733 & 89.24 & 0.035 & 4.44  \\
		%			\hline
		{} & 18 & 7.958 & 446.71 & - & - & - & - & 0.101 & 8.09  \\
		%			\hline
		{} & 24 & 23.448 & 692.30 & - & - & - & - & 0.252 & 14.57  \\
		%			\hline
		{} & 30 & 48.800 & 1026.48 & - & - & - & - & 0.509 & 23.61  \\
		%			\hline
		{} & 36 & 105.183 & 1619.01 & - & - & - & - & 1.005 & 50.49  \\
		%			\hline
		{} & 42 & - & - & - & - & - & - & 1.791 & 57.93  \\
		%			\hline
		{} & 48 & - & - & - & - & - & - & 2.679 & 86.67  \\
		%			\hline
		{} & 51 & - & - & - & - & - & - & 3.453 & 129.83  \\
		\hline
		\multirow{6}{*}{$P_3$} & 6 & 0.331 & 322.02 & 0.089 & 9.04 & 0.033 & 21.27 & 0.012 & 3.03  \\
		%			\hline
		{} & 12 & 2.059 & 322.07 & 22.749 & 76.91 & 21.897 & 89.22 & 0.035 & 4.93  \\
		%			\hline
		{} & 18 & 7.995 & 449.13 & - & - & - & - & 0.110 & 9.59  \\
		%			\hline
		{} & 24 & 23.578 & 696.74 & - & - & - & - & 0.286 & 21.04  \\
		%			\hline
		{} & 30 & 51.774 & 1138.27 & - & - & - & - & 0.643 & 30.09  \\
		%			\hline
		{} & 36 & 106.027 & 1628.84 & - & - & - & - & 1.287 & 66.14  \\
		%			\hline
		{} & 42 & - & - & - & - & - & - & 2.138 & 86.29  \\
		%			\hline
		{} & 48 & - & - & - & - & - & - & 3.369 & 170.94  \\
		%			\hline
		{} & 51 & - & - & - & - & - & - & 4.333 & 149.03  \\
		\hline
		\multirow{6}{*}{$P_4$} & 6 & 0.446 & 321.97 & 0.089 & 9.04 & 0.033 & 21.27 & 0.039 & 3.38  \\
		%			\hline
		{} & 12 & 8.289 & 552.62 & 22.749 & 76.91 & 21.897 & 89.22 & 150.115 & 986.64  \\
		%			\hline
		{} & 18 & - & - & - & - & - & - & - & -  \\
		%			\hline
		{} & 24 & - & - & - & - & - & - & - & -  \\
		%			\hline
		{} & 30 & - & - & - & - & - & - & - & -  \\
		%			\hline
		{} & 36 & - & - & - & - & - & - & - & -  \\
		%			\hline
		{} & 42 & - & - & - & - & - & - & - & -  \\
		%			\hline
		{} & 48 & - & - & - & - & - & - & - & -  \\
		%			\hline
		{} & 51 & - & - & - & - & - & - & - & -  \\
		\hline
		\multirow{6}{*}{$P_5$} & 6 & 0.430 & 322.03 & 0.031 & 9.09 & 0.047 & 21.19 & 0.011 & 3.10  \\
		%			\hline
		{} & 12 & 3.398 & 363.78 & 22.747 & 77.01 & 22.029 & 89.17 & 0.040 & 4.81  \\
		%			\hline
		{} & 18 & 18.176 & 783.24 & - & - & - & - & 0.115 & 10.99  \\
		%			\hline
		{} & 24 & 87.432 & 2382.82 & - & - & - & - & 0.322 & 18.68  \\
		%			\hline
		{} & 30 & - & - & - & - & - & - & 1.414 & 47.68  \\
		%			\hline
		{} & 36 & - & - & - & - & - & - & 1.287 & 66.35  \\
		%			\hline
		{} & 42 & - & - & - & - & - & - & 2.405 & 142.86  \\
		%			\hline
		{} & 48 & - & - & - & - & - & - & 4.848 & 225.55  \\
		%			\hline
		{} & 51 & - & - & - & - & - & - & 5.177 & 225.66  \\
		\hline
	\end{tabular}
	\tabcaption{测试集三中互斥算法测试用例的实验数据}
	\label{tabl:data:mutual}
\end{figure}

\begin{figure}[h!]\scriptsize
	\centering
	\setlength{\tabcolsep}{8pt}
	\renewcommand\arraystretch{0.8}
	\begin{tabular}{| r | r | r | r | r | r | r | r | r | r |}
		\hline
		\textbf{性质} & \textbf{进程数} & \multicolumn{8}{c|}{环算法} \\
		\hline
		{} & {} & \multicolumn{2}{c|}{\verds{}} & \multicolumn{2}{c|}{\nusmv{}} & \multicolumn{2}{c|}{\nuxmv{}} &  \multicolumn{2}{c|}{\sctl{}} \\
		\hline
		{} & {} & sec & MB & sec & MB & sec & MB & sec & MB \\
		\hline
		\multirow{6}{*}{$P_1$} & 3 & 0.168 & 322.09 & 0.040 & 10.02 & 0.045 & 22.08 & 4.622 & 62.22  \\
		%			\hline
		{} & 4 & 0.216 & 322.12 & 0.299 & 22.46 & 0.255 & 34.96 & - & -  \\
		%			\hline
		{} & 5 & 0.301 & 322.07 & 2.421 & 59.31 & 1.195 & 71.53 & - & -  \\
		%			\hline
		{} & 6 & 0.449 & 322.13 & 22.127 & 80.49 & 17.967 & 92.82 & - & -  \\
		%			\hline
		{} & 7 & 0.740 & 322.19 & 147.895 & 224.17 & 131.735 & 236.50 & - & -  \\
		%			\hline
		{} & 8 & 1.115 & 322.09 & 1135.882 & 865.04 & 1083.48 & 877.36 & - & -  \\
		%			\hline
		{} & 9 & 1.646 & 322.07 & - & - & - & - & - & -  \\
		%			\hline
		{} & 10 & 2.232 & 321.96 & - & - & - & - & - & -  \\
		\hline
		\multirow{6}{*}{$P_2$} & 3 & - & - & 0.058 & 10.74 & 0.068 & 22.73 & 0.031 & 3.22  \\
		%			\hline
		{} & 4 & - & - & 0.583 & 40.29 & 0.562 & 52.61 & 0.125 & 3.73  \\
		%			\hline
		{} & 5 & - & - & 5.164 & 62.29 & 5.295 & 74.62 & 0.444 & 4.05  \\
		%			\hline
		{} & 6 & - & - & 39.085 & 81.85 & 37.969 & 93.96 & 1.373 & 4.71  \\
		%			\hline
		{} & 7 & - & - & 246.123 & 229.07 & 241.375 & 241.15 & 3.745 & 6.03  \\
		%			\hline
		{} & 8 & - & - & - & - & - & - & 9.154 & 7.61  \\
		%			\hline
		{} & 9 & - & - & - & - & - & - & 19.997 & 10.07  \\
		%			\hline
		{} & 10 & - & - & - & - & - & - & 40.331 & 13.05  \\
		\hline
		\multirow{6}{*}{$P_3$} & 3 & - & - & 0.045 & 10.03 & 0.071 & 22.32 & 0.022 & 3.20  \\
		%			\hline
		{} & 4 & - & - & 0.296 & 22.46 & 0.299 & 34.96 & 0.820 & 13.11  \\
		%			\hline
		{} & 5 & - & - & 2.357 & 59.31 & 2.526 & 71.63 & 111.96 & 676.29  \\
		%			\hline
		{} & 6 & - & - & 22.147 & 80.49 & 21.304 & 92.93 & - & -  \\
		%			\hline
		{} & 7 & - & - & 147.567 & 224.17 & 141.134 & 236.74 & - & -  \\
		%			\hline
		{} & 8 & - & - & - & - & - & - & - & -  \\
		%			\hline
		{} & 9 & - & - & - & - & - & - & - & -  \\
		%			\hline
		{} & 10 & - & - & - & - & - & - & - & -  \\
		\hline
		\multirow{6}{*}{$P_4$} & 3 & 0.158 & 322.09 & 0.066 & 10.00 & 0.171 & 22.32 & 0.024 & 3.24  \\
		%			\hline
		{} & 4 & 0.190 & 322.05 & 0.356 & 22.46 & 0.367 & 34.95 & 0.104 & 3.82  \\
		%			\hline
		{} & 5 & 0.263 & 322.04 & 2.726 & 59.31 & 2.781 & 71.63 & 0.385 & 3.99  \\
		%			\hline
		{} & 6 & 0.385 & 322.07 & 27.013 & 80.48 & 24.794 & 94.95 & 1.289 & 4.57  \\
		%			\hline
		{} & 7 & 0.528 & 322.07 & 181.007 & 224.16 & 166.725 & 236.61 & 3.727 & 5.29  \\
		%			\hline
		{} & 8 & 0.815 & 322.14 & - & - & - & - & 9.525 & 7.14  \\
		%			\hline
		{} & 9 & 1.138 & 322.19 & - & - & - & - & 21.568 & 9.31  \\
		%			\hline
		{} & 10 & 1.574 & 321.98 & - & - & - & - & 45.097 & 12.95  \\
		\hline
	\end{tabular}
	%		\end{minipage}
	
	%	\end{figure}
	%	\end{tabular}
	\tabcaption{测试集三中环算法测试用例的实验数据}
	\label{tabl:data:ring}
\end{figure}

\begin{figure}[h!]
	\centering
	\scriptsize
	\setlength{\tabcolsep}{8pt}
	\begin{tabular}{| l | c | r | r | r | r |}
		\hline
		\multirow{2}{1.3cm}{\textbf{测试用例}} & \multirow{2}{1cm}{\textbf{有死锁}} & \multicolumn{2}{c|}{\sctlprov{}} & \multicolumn{2}{c|}{\CADP{}}\\
		\cline{3-6}
		{}&{} & sec & MB & sec & MB\\
		\hline
		vasy\_0\_1 & No & 0.13 &	27.16 &	0.40 &	10.95\\\hline
		cwi\_1\_2 & No & 0.13 &	27.67 &	0.39 &	10.80\\\hline
		vasy\_1\_4 	& No & 0.14 &	27.75 &	0.39 &	10.71\\\hline
		cwi\_3\_14 	& Yes & 0.14 &	25.70 &	0.40 &	10.82\\\hline
		vasy\_5\_9 	& Yes &	0.14 &	25.70 &	0.40 &	10.82\\\hline
		vasy\_8\_24 & No &	0.17 & 	28.90 &	0.43 &	10.79\\\hline
		vasy\_8\_38 & Yes &	0.14 &	25.76 &	0.39 &	10.74\\\hline
		vasy\_10\_56 &	No & 0.20 &	29.90 &	0.43 &	10.86\\\hline
		vasy\_18\_73 &	No & 0.24 &	31.68 &	0.47 &	11.81\\\hline
		vasy\_25\_25 &	Yes & 0.97 & 33.52 & 2.18 &	23.26\\\hline
		vasy\_40\_60 &	No &	0.21 &	29.42 &	0.46 &	15.08\\\hline
		vasy\_52\_318 &	No &	0.59 &	41.09 &	0.65 &	16.69\\\hline
		vasy\_65\_2621 & No &	1.41 &	77.02 &	2.09 &	109.03\\\hline
		vasy\_66\_1302 & No &	0.89 &	34.92 &	1.25 &  14.13\\\hline
		vasy\_69\_520 &	Yes &	0.23 &	27.47 &	0.51 &	11.84\\\hline
		vasy\_83\_325 &	Yes &	0.21 &	27.96 &	0.48 &	11.32\\\hline
		vasy\_116\_368 & No &	0.67 &	35.27 &	0.77 &	14.40\\\hline
		cwi\_142\_925 & Yes &	0.28 &	28.33 &	0.57 &	12.72\\\hline
		vasy\_157\_297 & Yes &	0.18 &	27.14 &	0.45 &	11.48\\\hline
		vasy\_164\_1619 & No &	2.53 &	48.39 &	1.53 &	22.90\\\hline
		vasy\_166\_651 & Yes &	0.29 &	31.19 &	0.55 &	13.30\\\hline
		cwi\_214\_684 &	Yes &	0.39 &	34.39 &	0.63 &	22.94\\\hline
		cwi\_371\_641 &	No &	1.36 &	40.41 &	1.24 &	42.92\\\hline
		vasy\_386\_1171 & No &	2.14 &	74.11 &	1.66 &	45.12\\\hline
		cwi\_566\_3984 & Yes &	0.78 &	38.53 &	1.11 &	21.92\\\hline
		vasy\_574\_13561 &	No & 18.23 &	246.97 &	9.72 &	188.21\\\hline
		vasy\_720\_390 	& Yes &	0.23 &	28.49 &	0.48 &	12.89\\\hline
		vasy\_1112\_5290 & No &	10.2 &	89.81 &	6.54 & 97.47\\\hline
		cwi\_2165\_8723 &	No &16.51 &	166.74 &14.55 &	185.58 \\\hline
		cwi\_2416\_17605 &	Yes &3.19 &	87.61 &	3.38 &	71.80 \\\hline
		vasy\_2581\_11442 &	Yes &2.40 &	74.11 &	2.68 &	58.43 \\\hline
		vasy\_4220\_13944 &	Yes &2.85 &	89.50 &	3.20 &	73.82 \\\hline
		vasy\_4338\_15666 &	Yes &3.41 &	96.21 &	3.83 &	80.59 \\\hline
		vasy\_6020\_19353 &	No &37.19 &	456.34 &	74.24 &	649.41 \\\hline
		vasy\_6120\_11031 &	Yes &2.35 &	82.57 	&2.60 &	67.01 \\\hline
		cwi\_7838\_59101 &	No &72.76 &	1013.67 &	140.21 &1019.55\\\hline
		vasy\_8082\_42933 &	No & 7.85 &	309.74 	&7.82 &	240.69 	\\\hline
		vasy\_11026\_24660 & Yes & 4.82 &149.80 &5.15 &	134.17 \\\hline
		vasy\_12323\_27667 	&Yes &5.40 &	164.73 & 5.67 &	149.09 \\\hline
		cwi\_33949\_165318 	&	No &	366.51 &	2368.22 &	636.39 &2972.61 
		\\\hline
		
		\hline
		
	\end{tabular}
	\tabcaption{\sctlprov{}与\CADP{}分别验证测试集四中测试用例的死锁性质的实验数据}
	\label{benchmark:vlts:deadlock}
	%\caption{}
\end{figure}


\begin{figure}[h!]
	\centering
	\scriptsize
	\setlength{\tabcolsep}{8pt}
	\begin{tabular}{| l | c | r | r | r | r |}
		\hline
		\multirow{2}{1.3cm}{\textbf{测试用例}} & \multirow{2}{1cm}{\textbf{有活锁}} & \multicolumn{2}{c|}{\sctlprov{}} & \multicolumn{2}{c|}{\CADP{}}\\
		\cline{3-6}
		{}&{} & sec & MB & sec & MB\\
		\hline
		
		vasy\_0\_1 & No &0.13 &	27.45 &	0.48 &	14.57 \\\hline
		cwi\_1\_2 & No &0.14 &	27.74 &	0.50 &	14.61 \\\hline
		vasy\_1\_4 	& No &0.14& 	27.70& 	0.48 &	14.66 \\\hline
		cwi\_3\_14 	& No &0.16 &	28.18 &	0.50 &	14.57 \\\hline
		vasy\_5\_9 	& No &0.16 	&28.08 &	0.51 &	14.68 \\\hline
		vasy\_8\_24 & No &0.18 	&29.09 &	0.53 &	14.75 \\\hline
		vasy\_8\_38 & No &0.20 	&28.86 	&0.52 &	14.73 \\\hline
		vasy\_10\_56 &	No &0.23 &	30.33 &	0.55 &	15.34 \\\hline
		vasy\_18\_73 &	No &0.28 &	33.07 &	0.56 &	16.25 \\\hline
		vasy\_25\_25 &	No &0.96 &	33.54 &	2.25 &	27.84 \\\hline
		vasy\_40\_60 &	No &0.26 &	30.23 &	0.57 &	19.65 \\\hline
		vasy\_52\_318 &	Yes &0.21 &	27.66 &	0.55 &	15.45 \\\hline
		vasy\_65\_2621 & No &5.31 &	280.57 &	1.98 &	113.40 \\\hline
		vasy\_66\_1302 & No &2.40 &	38.33 &	1.30 &	18.50 \\\hline
		vasy\_69\_520 &	No &1.04 &	33.41 &	0.85 &	16.39 \\\hline
		vasy\_83\_325 &	No &0.83 &	39.27 &	0.76 &	19.40 \\\hline
		vasy\_116\_368 & No &1.19 &	42.14 &	0.86 &	15.74 \\\hline
		cwi\_142\_925 & No &2.67 &	46.30 &	1.22 &	17.89 \\\hline
		vasy\_157\_297 & No &0.80 &	33.99 &	0.78 &	17.91 \\\hline
		vasy\_164\_1619 & No &3.69 &	53.30 &	1.51& 	23.44 \\\hline
		vasy\_166\_651 & No &1.60 	&49.98 &	1.02 &	26.02 \\\hline
		cwi\_214\_684 &	Yes &0.26 	&29.93 &	0.63 &	16.81 \\\hline
		cwi\_371\_641 &	Yes &0.26 	&30.63 &	0.62 &	17.41 \\\hline
		vasy\_386\_1171 & No &2.91 	&80.16 &	1.55 &	41.75 \\\hline
		cwi\_566\_3984 & No &13.32 	&106.25 &	3.95 &	54.08 \\\hline
		vasy\_574\_13561 &	No &27.02& 	272.11 &	8.17 &	188.69 \\\hline
		vasy\_720\_390 	& No &	0.86 &	31.45 	&0.76 &	17.45 \\\hline
		vasy\_1112\_5290 & No &10.49 &	89.86 &	5.24 &	97.93 \\\hline
		cwi\_2165\_8723 &	Yes &1.87 &	61.94 &	2.15 &	48.80 \\\hline
		cwi\_2416\_17605 &	Yes &3.10 &	87.61 &	3.44 &	76.30 \\\hline
		vasy\_2581\_11442 &	No &32.70 &	326.38 &	14.78& 	214.93 \\\hline
		vasy\_4220\_13944 &	No &43.93 &	423.03 	&24.71 &	330.85 \\\hline
		vasy\_4338\_15666 &	No &47.55 &	479.15 	&28.06 &	344.64 \\\hline
		vasy\_6020\_19353 &	Yes &3.23 &	100.24 	&3.57 &	106.43 \\\hline
		vasy\_6120\_11031 &	No &30.59 &	425.64 	&38.37 &	437.71 \\\hline
		cwi\_7838\_59101 &	Yes &11.34 &	250.68 &	11.58 &	236.09 \\\hline
		vasy\_8082\_42933 &	No & 119.77 &	1123.85 &	106.49& 	908.29 \\\hline
		vasy\_11026\_24660 &	No &60.86& 	698.85 &	108.97 &	804.34 \\\hline
		vasy\_12323\_27667 	&	No &68.50 &	793.83 &	134.44 	&898.61 \\\hline
		cwi\_33949\_165318 	&	Yes &33.89 &	732.05& 	34.60& 	738.78 \\\hline
	\end{tabular}
	\tabcaption{\sctlprov{}与\CADP{}分别验证测试集四中测试用例的活锁性质的实验数据}
	\label{benchmark:vlts:livelock}
\end{figure}

