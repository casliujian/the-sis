%%
%%% >>> Title Page
%%
%%
%%% Chinese Title Page
%%
  \confidential{}% show confidential tag
  \schoollogo{scale=4.2}{ucas_logo}% university logo
  \title[基于模型检测和定理证明的形式化验证:基础理论与相关工具]{基于模型检测和定理证明的形式化验证:基础理论与相关工具}% \title[short title for headers]{Long title of thesis}
  \author{刘坚}% name of author
  \advisor{蒋颖~研究员}% names and titles of supervisors
  \advisorsec{中国科学院软件研究所}% institute names of supervisors
  \degree{博士}% degree
  \degreetype{工学}% degree type
  \major{计算机软件与理论}% major
  \institute{中国科学院软件研究所}% institute of author
  %\chinesedate{2014~年~06~月}% only need for user customized date
%%
%%% English Title Page
%%
  \englishtitle{Towards Combing Model Checking and Theorem Proving: \\Theory and Related Tools}
  \englishauthor{Jian Liu}
  \englishadvisor{Professor Ying Jiang}
  \englishdegree{Doctor}
  \englishthesistype{thesis}
  \englishmajor{Computer Software and Theory}
  \englishinstitute{Institute of Software, Chinese Academy of Sciences}
  %\englishdate{June, 2014}% only need for user customized date
%%
%%% Generate Chinese Title
%%
\maketitle
%%
%%% Generate English Title
%%
\makeenglishtitle
%%
%%% >>> Author's declaration
%%
\makedeclaration
%%
%%% >>> Abstract
%%
\chapter{摘\quad 要}% does not show the title on the top
%\begin{abstract}% will show the title on the top
模型检测和逻辑推演是目前用于形式化验证系统正确性的主要的两种方法。模型检测的优点是可以实现完全自动化,验证过程不需要人工干预,缺点是在处理大型系统或者无穷状态系统的过程中通常会面临状态爆炸问题;逻辑推演的优点是通常不关心系统的状态,从而避免了状态爆炸问题的产生,而缺点则是通常不能实现完全自动化, 在验证的过程中通常需要人工干预。因此,如何结合模型检测和逻辑推演两种方法的优点来建立新的形式化验证方法是本文的主要研究内容。

本文的第一个工作是,建立一个逻辑系统\CTLP{},\CTLP{}以Kripke模型为参数,对于一个给定的Kripke模型$\cal M$,一个\CTLP$(\mathcal{M})$公式是有效的当且仅当这个公式在M中是满足的。相比于在\CTL{}逻辑中只能讨论模型中当前状态的性质,在\CTLP{}中我们可以定义模型中的状态之间的关系,从而丰富了\CTL{}逻辑的表达性;然后,我们根据逻辑\CTLP{}建立了一个证明系统\SCTL{}(Sequent-calculus-like proof system for \CTLP{}),并证明了\SCTL{}系统的可靠性和完备性,使得一个公式在\SCTL{}中是可证的当且仅当它在给定的模型中是满足的。

本文的第二个工作是,提出了一个\SCTL{}证明系统的工具实现\sctlprov{},\sctlprov{}既可以看作为定理证明器也可看作为模型检测工具:相比于定理证明器,\sctlprov{}可以应用更多的优化策略,比如利用\BDD{}(Binary Decision Diagram)来存储状态集合从而减小内存占用;相比于模型检测工具,\sctlprov{}的输出是完整的证明树,比模型检测工具的输出更丰富。

本文的第三个工作是,实现了一个3D证明可视化工具\textsf{VMDV}(Visualization for Modeling, Demonstration, and Verification)。\textsf{VMDV}目前可以完整的显示\sctlprov{}的证明树以及高亮显示\sctlprov{}的证明过程。同时\textsf{VMDV}是一个一般化的证明可视化工具,并提供了不同的接口用来与不同的定理证明器(比如coq)协同工作。


\keywords{模型检测,定理证明,工具实现}
%\end{abstract}


\chapter{Abstract}% does not show the title on the top
%\begin{englishabstract}% will show the title on the top
This paper is a help documentation for the \LaTeX{} class ucasthesis, which is  a thesis template for the University of Chinese Academy of Sciences. The main content is about how to use the ucasthesis, as well as how to write thesis efficiently by using \LaTeX{}.

\englishkeywords{University of Chinese Academy of Sciences (UCAS), Thesis, \LaTeX{} Template}
%\end{englishabstract}
