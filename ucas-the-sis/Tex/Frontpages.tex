%---------------------------------------------------------------------------%
%->> 封面信息及生成
%---------------------------------------------------------------------------%
%-
%-> 中文封面信息
%-
%\confidential{}% 密级:只有涉密论文才填写
%\schoollogo{scale=0.095}{ucas_logo}% 校徽
%\title{中国科学院大学学位论文\LaTeX{}模板 {$~^{\pi}\pi^{\pi}$}}% 论文中文题目
%\author{莫晃锐}% 论文作者
%\advisor{刘青泉~研究员~中国科学院力学研究所}% 指导教师:姓名 专业技术职务 工作单位
%\advisorsec{}% 第二指导老师:按情况填写
%\degree{硕士}% 学位:学士、硕士、博士
%\degreetype{理学}% 学位类别:理学、工学、工程、医学等
%\major{流体力学}% 二级学科专业名称
%\institute{中国科学院力学研究所}% 院系名称
  \confidential{}% show confidential tag
\schoollogo{scale=0.095}{ucas_logo}% university logo
\title[基于模型检测和定理证明的形式化验证:基础理论与相关工具]{基于模型检测和定理证明的形式化验证:基础理论与相关工具}% \title[short title for headers]{Long title of thesis}
\author{刘坚}% name of author
\advisor{蒋颖~研究员}% names and titles of supervisors
\advisorsec{中国科学院软件研究所}% institute names of supervisors
\degree{博士}% degree
\degreetype{工学}% degree type
\major{计算机软件与理论}% major
\institute{中国科学院软件研究所}% institute of author
\chinesedate{2018~年~6~月}% 毕业日期:夏季为6月、冬季为12月
%-
%-> 英文封面信息
%-
\englishtitle{Towards Combing Model Checking and Theorem Proving: \\Theory and Related Tools}% 论文英文题目
\englishauthor{Jian Liu}% 论文作者
\englishadvisor{Supervisor: Professor Ying Jiang}% 指导教师
\englishdegree{Doctor}% 学位:Bachelor, Master, Doctor。封面格式将根据英文学位名称自动切换,请确保拼写准确无误
\englishdegreetype{Philosophy}% 学位类别:Philosophy, Natural Science, Engineering, Economics, Agriculture 等
\englishthesistype{dissertation}% 论文类型: thesis, dissertation
\englishmajor{Computer Software and Theory}% 二级学科专业名称
\englishinstitute{Institute of Software, Chinese Academy of Sciences}% 院系名称
\englishdate{June, 2018}% 毕业日期:夏季为June、冬季为December
%-
%-> 生成封面
%-
\maketitle% 生成中文封面
\makeenglishtitle% 生成英文封面
%-
%-> 作者声明
%-
\makedeclaration% 生成声明页
%-
%-> 中文摘要
%-
\chapter*{摘\quad 要}\chaptermark{摘\quad 要}% 摘要标题
\setcounter{page}{1}% 开始页码
\pagenumbering{Roman}% 页码符号

模型检测和逻辑推演是目前用于形式化验证系统正确性的主要的两种方法。模型检测的优点是可以实现完全自动化,验证过程不需要人工干预,缺点是在处理大型系统或者无穷状态系统的过程中会面临状态爆炸问题;逻辑推演的优点是通常不关心系统的状态,从而避免了状态爆炸问题的产生,而缺点则是通常不能实现完全自动化,在验证的过程中需要人工干预。因此,如何结合模型检测和逻辑推演两种方法的优点来建立新的形式化验证方法是本文的主要研究内容。

本文的第一个工作是,建立一个逻辑系统\CTLP{},\CTLP{}以Kripke模型为参数,对于一个给定的Kripke模型$\cal M$,一个\CTLP$(\mathcal{M})$公式是有效的当且仅当这个公式在M中是满足的。相比于在\CTL{}逻辑中只能讨论模型中当前状态的性质,在\CTLP{}中我们可以定义模型中的状态之间的关系,从而丰富了\CTL{}逻辑的表达性;然后,我们根据逻辑\CTLP{}建立了一个证明系统\SCTL{}(Sequent-calculus-like proof system for \CTLP{}),并证明了\SCTL{}系统的可靠性和完备性,使得一个公式在\SCTL{}中是可证的当且仅当它在给定的模型中是满足的。

本文的第二个工作是,提出了一个\SCTL{}证明系统的工具实现\sctlprov{},\sctlprov{}既可以看作为定理证明器也可看作为模型检测工具:相比于定理证明器,\sctlprov{}可以应用更多的优化策略,比如利用\BDD{}(Binary Decision Diagram)来存储状态集合从而减小内存占用;相比于模型检测工具,\sctlprov{}的输出是完整的证明树,比模型检测工具的输出更丰富。

本文的第三个工作是,实现了一个3D证明可视化工具\textsf{VMDV}(Visualization for Modeling, Demonstration, and Verification)。 目前,\textsf{VMDV}可以完整的显示定理证明工具 \sctlprov{} 的输出(证明树以及Kripke模型)以及高亮显示\sctlprov{}的证明过程。与此同时,\textsf{VMDV}是一个一般化的证明可视化工具,并提供了一个统一的接口用来与不同的定理证明器(比如\tool{Coq})协同工作。


\keywords{形式化验证,模型检测,定理证明,工具实现}
%-
%-> 英文摘要
%-
\chapter*{Abstract}\chaptermark{Abstract}% 摘要标题

Model checking and automated theorem proving are two pillars of formal methods. The advantage of model checking is its complete automaticity, while the main obstacle is the state-explosion problem. Theorem proving usually does not enumerate the states of a system, thus does not have the state-explosion problem. However, theorem proving usually cannot be fully automatic. The main content of this thesis is to investigate model checking from an automated theorem proving perspective, aiming at combining the expressiveness of automated theorem proving and the complete automaticity of model checking.

The first work of the thesis is to build a logic \CTLP{}, which is a slight extension of the Computation Tree Logic (\CTL{}). Then, we build a sequent-calculus-like proof system for \CTLP{}, called \SCTL{}. \SCTL{} is parameterized by a Kripke model, and unlike the usual sequent calculus, a \CTLP{} formula is provable in \SCTL($\cal M$) if and only if the formula is valid in the given Kripke model $\cal M$. The soundness and completeness of \SCTL{} are also proved. In order to search proofs efficiently, we present a proof search method for \SCTL{}, which is based on the notion of continuation, and a proof search algorithm based on the proof search method.


The second work of the thesis is the implementation of a theorem prover, called \sctlprov{}. For a given input file, which consists of a Kripke model and its properties specified, \sctlprov{} searches for a proof for each of the properties for the given Kripke model using the proof search algorithm presented. 
\sctlprov{} can be seen as either a theorem prover, or a model checker. 
Comparing with traditional theorem provers, \sctlprov{} can use more optimization strategies in the model checking of \CTL{} properties, such as using \BDD{} to store visited states to reduce space occupation; comparing with traditional model checkers, the output of \sctlprov{} is much more expressive: 
instead of expressing counterexamples as sequences of states when the verified property does not hold, \sctlprov{} outputs a proof tree of the given property when the verification succeeds, and a proof tree of the negation of the given property otherwise.

The third work of the thesis is an implementation of a 3D visualization tool, called \tool{VMDV}. Using \tool{VMDV}, we can visualize the proof produced by \sctlprov{} and the Kripke model for a better comprehension of the verification results. Furthermore, \tool{VMDV} provides a generalized protocol for integration with various kinds of theorem provers, such as \tool{Coq}.


%\keywords{形式化验证;模型检测;定理证明;工具实现}
\englishkeywords{Formal Verification, Model Checking, Theorem Proving, Implementation}% 英文关键词
%---------------------------------------------------------------------------%
