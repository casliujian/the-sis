\chapter{引言}\label{chapt:intro}
\section{选题背景和意义}
近年来,计算机软硬件系统的飞速发展极大地提高了人类的生产效率,也极大地方便了人类的生活。但与此同时,由于软硬件故障而导致的人身以及生产事故同样不可忽视,
尤其是对于安全攸关的系统,比如航空航天系统、高速铁路控制系统、医疗卫生系统等,这些系统的故障往往会导致灾难性的后果。比如,1994年,Intel公司由于其发售的奔腾系列处理器的指令集出现漏洞而宣布召回所有缺陷产品,并因此造成4.75亿美元的经济损失。1996年,由于飞行控制程序中存在由64位浮点数到16位整型的错误转换,阿丽亚娜5号运载火箭在首次发射中仅仅飞行37秒就发生爆炸,并造成有史以来损失最惨重的航天事故。因此,随着软硬件系统的复杂度的提高,如何确保系统的正确性成为系统开发面临的关键性问题之一。

传统的计算机软硬件系统的开发一般依靠仿真和测试的方法来评估系统是否满足设计要求。基于仿真和测试的方法在系统开发的初始阶段可以很大地提高开发效率,并便于即时调试。随着技术的不断发展和完善,此类方法已经可以越来越方便地发现系统设计的一些隐匿的错误。由于其高效性,现如今的大量软硬件系统的开发过程依然采用此类方法。然而,基于仿真和测试的方法往往无法确保找到系统的所有错误,因而无法保证系统的正确性。

除了基于仿真和测试的方法,对系统正确性的评估还可通过形式化验证来进行。与基于仿真和测试的方法不同,通过形式化方法验证过的系统通常可确保其正确性。在利用形式化方法验证系统性质的过程中,系统的所有状态或行为通常用形式化描述语言来定义,然后分别利用数学的方法枚举由形式语言所定义的所有的系统状态,直到得出验证结论。
\section{形式化方法}
在众多的形式化验证方法中,模型检测和定理证明是最被普遍应用的两种。
\paragraph{模型检测}
20世纪80年代,模型检测\cite{CGP01,BouajjaniJNT00,BaierKatoen08}最早由Clark和Emerson等人提出,并很快得到迅猛发展。
在模型检测中,被验证的系统通常被抽象为一个有限状态模型,模型中的有限个状态对应于系统的状态,同时模型中的状态迁移通常用来描述系统在运行过程中的状态变化。系统的性质通常由时序逻辑公式来描述。在模型检测中,验证系统的性质通常指的是判定时序逻辑公式在有限状态模型上的可满足性。模型检测工具通常会通过遍历状态空间来验证公式的可满足性。当需要验证的公式是可满足的时候,模型检测工具会返回true;反之,当需要验证的公式不可满足时,模型检测工具会产生一个反例用来定位系统的错误。近年来,随着技术和理论的不断发展,一大批优秀的模型检测工具不断地涌现出来,其中被广泛应用的模型检测工具以符号模型检测工具\nusmv{}\cite{CimattiCGR99}、限界模型检测工具\verds{}\cite{Zhang14}、即时模型检测工具\tool{SPIN}\cite{Holzmann97}以及形式化验证工具集\CADP{}\cite{GaravelLMS13}为代表。
\begin{itemize}
	\item \nusmv{}:\nusmv{}的提出是基于最早由McMillan在其博士论文\cite{mcmillan93}中提出的符号模型检测方法。符号模型检测主要应用于对计算树逻辑(Computation Tree Logic,简称\CTL{})公式的验证,符号模型检测最大的特点在于将要验证的性质看成系统状态的集合,而状态集合通常用\BDD{}\cite{Bryant86}来进行表示。一个\CTL{}公式的可满足行则表示为系统的初始状态是否属于该状态集合。在系统的验证过程中,系统的状态空间不是显示地生成出来,而是由\BDD{}来符号化地表示,这种状态集合的表示方式在表达相同数量的状态时相比显示表示每个状态的方式通常占用更少的空间。
	\item \verds{}:
	限界模型检测工具\verds{}由Zhang\cite{Zhang14}提出,不同于符号模型检测工具,\verds{}在验证公式的过程中首先规定一个界,然后在界内搜索模型的状态。在实际的验证过程中,相比于\nusmv{},\verds{}通常只需访问较少的状态就能完成公式的验证。另外,不同于基于\textsf{SAT}\cite{BCCZ99}的限界模型检测工具,\verds{}在验证过程中将时序逻辑公式翻译到一个\textsf{QBF}(Quantified Boolean Formula),相比于\textsf{SAT}公式,\textsf{QBF}可用来表示更加复杂的时序逻辑公式。
	\item \tool{SPIN}:\tool{SPIN}是在1980年由美国贝尔实验室的Holzmann等人编写,并在1991年免费开放的一种即时模型检测工具。不同于\nusmv{}和\verds{},\tool{SPIN}更多的时候被用来验证分布式系统模型的线性时序逻辑(Linear Temporal Logic,简称\textsf{LTL})性质。在\tool{SPIN}的验证过程不需要提前规定模型搜索的界,而是通过深度优先的方式逐个访问模型的状态,直到可以计算出\textsf{LTL}公式的可满足性为止。另外,不同于\nusmv{}和\verds{},模型的状态在\tool{SPIN}中通常显式存储,不过由于需要访问的状态通常比前两种工具少,因此在某些模型的验证中空间占用比前两种工具更少。
	\item \tool{CADP}:\tool{CADP}全称Construction and Analysis of Distributed Processes,是一系列用于通讯协议和分布式系统的设计和验证的工具的集合。\tool{CADP}目前由法国国家信息与自动化研究所的\textsf{CONVECS}小组开发(起初由\textsf{VASY}小组开发),并在工业界有广泛的应用。不同于验证Kripke模型的工具,\textsf{CADP}主要用来验证基于动作的形式化系统,比如符号迁移系统、马尔可夫链等。\CADP{}工具集中包含可验证多种时序逻辑公式的工具,并可以采用即时模型检测或符号模型检测的状态搜索策略。
\end{itemize}

模型检测的主要优点是验证过程完全自动化,主要缺点是状态爆炸问题,尤其是在对多进程的并发系统的验证过程中,随着进程数量的增长,系统状态的数量可能呈现指数级的增长。状态爆炸问题是困扰模型检测在大型系统的验证中应用的主要问题。

\paragraph{定理证明}
不同于基于状态的模型检测方法,定理证明\cite{Fitting96,Loveland78,Burel09}是基于证明的。在利用定理证明方法形式化验证系统的性质的时候,被验证的系统通常在一种数学语言中被表述为一系列数学定义,而要验证的性质则被表述为数学逻辑公式,然后从公理出发,利用推理规则寻找逻辑公式的形式化证明。除了验证过程,定理证明的验证结果的表示也通常与模型检测方法不同:定理证明的验证结果通常是一棵证明树,通常比模型检测的验证结果包含更详细的信息。在应用定理证明方法进行形式化验证时,定理证明工具的产生极大的增强了证明的效率。根据基于的数学逻辑的不同,不同的定理证明工具可大体被分为以下几类:
\begin{itemize}

	\item \textbf{经典高阶逻辑}:经典高阶逻辑的表达能力较强,通常可方便地用来描述复杂地系统行为和性质。然而,较好地表达性也意味着验证地自动化程度较低。基于经典高阶逻辑的代表性定理证明工具有\tool{HOL}\cite{Gordon00}、\tool{Isabelle/HOL}\cite{Nipkow12}、\tool{PVS}\cite{OwreRS92}等。
	\item \textbf{构造逻辑}:由柯里-霍华德同构可知,一个公式地构造性证明一一对应于一个可执行的计算机算法。因此,在自动化程度上,基于构造逻辑的定理证明工具相比基于经典高阶逻辑的工具更好。不过,公式的构造性证明往往比较难以找到。基于构造逻辑的代表性定理证明工具有\tool{Coq}\cite{BertotC04}、\tool{Nuprl}\cite{AllenCEKL00}等。
	\item \textbf{一阶逻辑}:在表达能力上,基于一阶逻辑的定理证明工具通常可以绝大多数系统的验证,同时此类定理证明器的自动化程度较高。基于一阶逻辑的代表性定理证明工具有\tool{ACL2}\cite{KaufmannM08}、\tool{Isabelle/FOL}\cite{cs-LO-9301106}等。
\end{itemize}
定理证明的优点在于表达能力强,而且验证过程不是基于状态的,因此可以被用来验证大型系统,甚至是无穷状态系统。不过,相比于模型检测,定理证明的缺点就是证明过程通常需要人工干预,而不是完全自动化的。

\section{相关工作}
模型检测和定理证明的优缺点是互补的:模型检测是完全自动化的,而定理证明通常是半自动化的;定理证明通常能对复杂的或者具有无穷状态的系统进行建模和验证,而模型检测通常无法验证大型或者无穷状态系统。因此,如何将这两种形式化验证方法相结合成为近年来的热门研究课题。截至目前,国际上已经有多个试图结合两种验证方法的工作。
\begin{itemize}
	\item Rajan,Shankar和Srivas提出了一种基于\tool{PVS}的模型检测和定理证明的结合方法\cite{RajanSS95}。在这种方法中,作者首先在\tool{PVS}中定义了一个$\mu$-演算理论,并用$\mu$-演算定义了时序逻辑\textsf{CTL}的模态词,当证明\textsf{CTL}公式的时候,\tool{PVS}则会调用调用内置的模型检测命令来完成证明。在这种方法中,模型检测方法被用来完成\tool{PVS}证明的子目标,而不是贯穿于整个证明过程。这种将模型检测和定理证明相结合的方式最大的优点是利用了定理证明器强大的表达和抽象能力将系统抽象成一个有穷状态模型,然后利用模型检测算法在有穷状态模型中完成状态搜索。
	\item 在证明\textsf{PLTL}公式方面,Cavalli和Cerro给出了一种基于经典的resolution的方法\cite{CavalliC84}。在这种方法中,作者将\textsf{PLTL}公式翻译到一种时序逻辑的合取范式,然后定义了一组这种合取范式的resolution规则。与这种方法类似,Venkatesh提出了一种基于resolution的判定\textsf{LTL}公式可满足性的方法\cite{Venkatesh85}。这种方法的特点是将给定的\textsf{LTL}公式转换成与该公式的可满足性等价的带有嵌套模态词的公式,然后在转换后的公式上应用经典的resolution规则。同样基于resolution的方法还有Fisher,Dixon和Peim提出的证明\textsf{LTL}公式有效性的方法\cite{FisherDP01},以及Zhang,Hustadt和Dixon提出的证明\textsf{CTL}公式有效性的方法\cite{ZhangHD14}等。
	\item Ji提出了一种基于resolution modulo方法的用于验证\textsf{CTL}公式的有效性的方法\cite{Ji15}。Resolution modulo方法首次由Dowek,Hardin和Kirchner提出\cite{DowekHK03,Dowek10},基本思想是将一阶逻辑的resolution方法转换成重写系统,因此公式的推导过程更多的用计算来实现。Ji的方法已经在定理证明器\tool{iProver Modulo}中实现,并在与现有的模型检测工具(\nusmv{}和\verds{})的实验结果对比中表现良好。
\end{itemize}
\section{本文的贡献和组织}
本文的主要内容是研究模型检测和定理证明两种方法的结合,并不同于已有的工作,本文既没有将模型检测作为定理证明的一个判定过程,也没有将已有时序逻辑公式的模型检测问题转化到定理证明问题,而是提出基于传统的时序逻辑的一个扩展,并在保持模型检测完全自动化的优点的基础上可以验证更复杂的性质。本文的工作基于我的导师蒋颖教授的想法,并在蒋颖教授的指导下完成。本文的主要工作如下:
\begin{enumerate}
	\item 建立一个针对\CTL{}逻辑的扩展:\CTLP{}。相比于\CTL{},在\CTLP{}中可以定义多元谓词。除了可以定义状态的时序性质之外,利用多元谓词可以表达某些“空间”性质,即多个状态之间的关系。针对\CTLP{},我们还定义了一个证明系统\SCTL{}用来表达\CTLP{}公式的证明,并证明了该证明系统的可靠性和完备性。除此之外,我们还提出了一种针对\SCTL{}系统的证明搜索算法并证明了该算法的终止性和正确性。
	\item 给出了一个对证明系统\SCTL{}的编程实现---定理证明工具\sctlprov{}。在不丢失效率的前提下,\sctlprov{}可被用来实现复杂系统的验证,证明结果相比于模型检测工具更丰富。为了对比\sctlprov{}与现有工具的效率,我们分别在若干个随机生成以及工业级测试用例集上与多个业界顶尖的形式化验证工具进行实验数据的对比。
	\item 设计并实现了定理证明可视化工具\tool{VMDV},并将\tool{VMDV}用于\sctlprov{}的证明输出的可视化。同时\tool{VMDV}还被设计并实现为一般化的证明可视化工具,并可实现不同的定理证明工具的输出的可视化。
\end{enumerate}

除本章外,本文其余章节的组织结构如下:

第二章介绍了\CTLP{}逻辑和证明系统\SCTL{},以及针对\SCTL{}的证明搜索算法。证明了\SCTL{}证明系统的可靠性和完备性,以及证明搜索算法的终止性和正确性。证明搜索算法的伪代码也在本章给出。

第三章介绍了定理证明工具\sctlprov{}的架构和实现细节,并在多个测试用例集上与多个形式化验证工具进行实验结果对比。同时给出\sctlprov{}验证复杂系统(模型检测工具无法处理)的一个案例的分析。

第四章介绍了定理证明可视化工具\tool{VMDV}的架构和实现细节。

第五章是全文总结和对未来工作的展望。