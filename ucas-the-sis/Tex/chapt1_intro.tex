\chapter{引言}\label{chapt:intro}
\section{选题背景和意义}
近些年以来,计算机软硬件系统的飞速发展极大地提高了人类的生产效率,也极大地方便了人类的生活。然而,近年来由于软硬件故障而导致的事故引起了人们的高度重视,尤其是对于安全攸关的系统,比如航空航天系统、高速铁路控制系统、医疗卫生系统等,这些系统的故障往往会导致灾难性的后果。比如,1994年,Intel公司由于其发售的奔腾系列处理器的指令集出现漏洞而宣布召回所有缺陷产品,并因此造成4.75亿美元的经济损失。1996年,由于飞行控制程序中存在由64位浮点数到16为整型的错误转换,阿丽亚娜5号运载火箭在首次发射中仅仅飞行37秒就发生爆炸,并造成有史以来损失最惨重的航天事故。因此,随着软硬件系统的复杂度的提高,如何确保系统的正确性成为系统开发面临的主要问题之一。

传统地计算机软硬件系统地开发依靠仿真和测试的方法来评估系统是否满足设计要求。基于仿真和测试的方法在系统开发的初始阶段可以很大的提高开发效率,并便于即时调试。而且随着技术的不断发展和完善,此类方法已经可以越来越方便地发现系统设计的一些隐匿的错误。由于其高效性,现如今的大量软硬件系统的开发过程依然采用此类方法。然而,基于仿真和测试的方法往往无法完全确保系统的正确性。

与基于仿真和测试的方法不同,形式化方法是一门基于数学的验证系统正确性的方法。在利用形式化方法验证系统性质的过程中,系统的所有状态或行为通常用形式化描述语言来定义,然后分别利用数学的方法分别枚举系统运行的所有情况,直到得出验证结论。
\section{形式化方法}
在众多的形式化验证方法中,模型检测和定理证明是最被普遍应用的两种。
\paragraph{模型检测}
20世纪80年代,模型检测\cite{CGP01,BouajjaniJNT00,BaierKatoen08}最早由Clark和Emerson等人提出,并很快得到迅猛发展。
在模型检测中,被验证的系统通常被抽象为一个有限状态模型,模型中的有限个状态,通常对应于系统运行的状态,同时模型中的状态迁移通常用来描述系统在运行过程中的状态变化。系统的性质通常由时序逻辑公式来描述。在模型检测中,验证系统的性质通常是计算时序逻辑公式在有限状态模型上的可满足性。在验证过程中,模型检测工具通常会通过遍历状态空间来验证公式的可满足性。当需要验证的公式是可满足的时候,模型检测工具通常会返回true;反之,当需要验证的公式不可满足时,模型检测工具通常会产生一个反例用来定位系统的错误。模型检测的验证过程是完全自动化的。近年来,随着技术和理论的不断发展,一大批优秀的模型检测工具不断地涌现出来,其中以符号模型检测工具\nusmv{}\cite{CimattiCGR99}、限界模型检测工具\verds{}\cite{Zhang14}、动态模型检测工具\tool{SPIN}\cite{Holzmann97}以及形式化验证工具集\CADP{}\cite{GaravelLMS13}为代表。
\begin{itemize}
	\item \nusmv{}:\nusmv{}基于最早由McMillan在其博士论文\cite{mcmillan93}中提出的符号模型检测方法。符号模型检测主要应用于对计算树逻辑(Computation Tree Logic,简称\CTL{})公式的验证,符号模型检测最大的特点在于将要验证的性质看成系统状态的集合,而且状态集合通常用\BDD{}\cite{Bryant86}来表示。一个\CTL{}公式的可满足行则表示为系统的初始状态是否属于该状态集合。在系统的验证过程中,系统的状态空间不是显示的生成出来,而是由\BDD{}来符号化地表示,这种状态集合的表示方式在表达相同数量的状态时相比显示表示每个状态的方式通常占用更少的空间。
	\item \verds{}:
	
	限界模型检测工具\verds{}由Zhang提出
\end{itemize}

模型检测的主要缺点是状态爆炸问题,尤其是在对多进程的并发系统的验证过程中,随着进程数量的增长,系统状态的数量可能呈现指数级的增长。状态爆炸问题是困扰模型检测在大型系统的验证中应用的主要问题。

\paragraph{定理证明}
\section{相关工作}
\section{本文的贡献和组织}