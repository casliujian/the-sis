
\chapter{公平性性质的验证}\label{chapt:fairness}

在并发系统的形式化验证中,公平性是比较重要的一个方面。在状态搜索中,公平性前提通常可以避免一些不切实际的无穷的路径,即忽略掉一些不切实际的系统行为\cite{BaierKatoen08}。然而,无论是\CTL{}还是\CTLP{}均无法在语法中表示公平性假设。这是由于在这两种逻辑中缺乏描述特定路径的性质的机制。即使如此,我们仍然可以基于上一章中介绍的证明搜索算法来完成带有公平性前提的\CTLP{}公式的验证。



\couic{性质是一类比较重要的性质。在验证公平性性质时,模型检测算法通常会规避系统的一些不切实际的行为\cite{BaierKatoen08},比如在两个进程的互斥模型中,我们通常只考虑没有进程饿死的情况,即没有永远处在等待状态的进程。在这类模型的验证中,我们就可以添加一个公平性限制来规避模型的不公平的迁移(即在该模型中的状态迁移过程中有进程饿死),从而只在对两个进程都公平的前提下进行状态迁移并验证模型的性质。}

在本章中,我们介绍\CTLP{}中公平性的定义,以及如何完成对公平性性质的验证。

\section{公平性}
在介绍公平性之前,让我们首先看如下的例子。

\begin{example} [进程调度的公平性]
	假设有一个具有$n+1$个进程的计算机系统,其中,进程$Server$为服务端进程,进程$P_1,...,P_n$为$n$个客户端进程。在系统的运行过程中,每个客户端进程可向服务端进程发出请求,而服务端进程则会根据收到的请求来为相应的客户端进程提供服务。该系统的一个可能的进程调度策略如下:系统从$P_1$到$P_n$依次检查是否有客户端进程发出请求,如果是,则$Server$为第一个系统找到的发出请求的客户端进程提供服务,当$Server$服务完成后,系统重复上述过程。
	
	在该系统中,如果客户端进程$P_1$持续向服务端进程$Server$发出请求,那么根据系统的进程调度策略,$Server$将持续为$P_1$提供服务,而忽略其他客户端进程的请求。这样的进程调度策略就被称为是\textit{不公平}的。在一个公平的进程调度策略中,当一个客户端进程发出请求后,服务端进程总会在有限时间内提供服务,而不会让发出请求的客户端进程无限等待。比如,在轮询调度(Round-Robin Scheduling)\cite{ASPBG14}中,系统按照轮询的顺序依次检查每个进程是否发出请求,而且当处理完一个进程的请求后,再按照顺序检查下一个进程。经过轮询调度的进程不会饿死(即无限等待),因此轮询调度是一种公平的进程调度策略。
\end{example}
本文中公平性的定义与McMillan提出的公平性定义\cite{mcmillan93}一致,即一个无穷路径是公平的指的是某种性质在该路径上无穷次成立。一个带有公平性前提的性质通常具有$E_Cf$或者$A_Cf$形式,其中$C$表示一个\CTLP{}公式的集合,而$f$表示路径的性质。在本文中,$E_CG(\phi)(s)$在给定的Kripke模型中成立当且仅当存在以$s$为起始状态的一个无穷路径,集合$C$中的每个公式在该路径上的无穷次成立,而且在该路径上的每个状态上$\phi$都成立,而且集合$C$中的每个公式$\phi'$,$\phi'$在$\pi$上的无穷个状态上都是成立的;$A_CF(\phi)(s)$在给定的Kripke模型中是有效的当且仅当对于所有以$s$为起始状态的无穷路径,集合$C$中的每个公式都在每条路径上无穷次成立,而且每条路径上都存在一个状态,使得$\phi$在该状态上成立。另外,与McMillan的定义类似,带有其他模态词的公平性性质可用$E_CG$与$A_CF$公式进行定义:


\begin{small}
	$$E_CX_x(\phi)(t) = EX_x(\phi \wedge E_CG_x(\top)(x))(t)$$
	$$A_CX_x(\phi)(t) = AX_x(\phi \vee A_CF_x(\bot)(x))(t)$$
	$$E_CU_{x,y}(\phi_1,\phi_2)(t) = EU_{x,y}(\phi_1, \phi_2\wedge E_CG_z(\top)(y))(t)$$
	$$A_CR_{x,y}(\phi_1,\phi_2)(t) = AR_{x,y}(\phi_1, \phi_2\vee A_CF_z(\bot)(y))(t)$$
\end{small}

\section{验证带有公平性假设的公式}
根据上一节的内容可知,要验证公式$E_CG_x(\phi)(s)$,只需找到一条以$s$为起始状态的公平的无穷路径使得$\phi$在这条路径上永远成立;要验证公式$A_CF(\phi)(s)$,只需确定不存在一条以$s$为其实状态的无穷路径使得$\phi$在这条路径上永远不成立。因此,要利用\SCTL{}的证明搜索策略来验证公平性性质,只需要一个机制来判断公平的无穷路径的存在与否。
\couic{Given that \SCTL{} is sound and complete, to prove $E_CG_x(\phi)(t)$ is equivalent to prove $EG_x(\phi)(t)$ where only fair paths are considered, i.e., to prove the existence of a fair path on which $\phi$ is always provable. Similarly, to prove $A_CF_x(\phi)(t)$ is equivalent to prove $AF_x(\phi)(t)$ where only fair paths are considered, i.e., to prove the absence of a fair path on which $\phi$ is always not provable. Thus, to prove \SCTL{} formulae with fairness constraints, we need a mechanism to decide the existence of fair paths.}

由命题\ref{prop:fair_if}与命题\ref{prop:fair_fi}可知,我们可以在证明搜索中利用merge来判断公平的无穷路径的存在与否:存在一条具有公平性前提$C$的无穷路径当且仅当存在一个merge,使得$C$中的每个公式都在该merge上的环中的某个状态上成立。



\begin{proposition}\label{prop:fair_if}
	令$C$为一个\CTLP{}公式的集合,而且$\sigma = s_0,s_1,...$是一条无穷路径,其中对于任意$i\ge 0$都有$s_i \rightarrow s_{i+1}$。如果$C$中的每个公式都在$\sigma$中无穷次成立,那么一定存在一条有穷路径$\sigma_f = s'_0,s'_1,...,s'_n$,使得对任意$0\le i\le n-1$都有$s'_j\rightarrow s'_{j+1}$而且存在$0\le p\le n-1$使得$s'_n = s'_p$,每个状态$s'_j$都在$\sigma$出现,而且对于$C$中的任意公式,都存在某个状态$s'_q$使得该公式成立,其中$p\le q\le n$。
\end{proposition}
\begin{proof}
	由于Kripke模型中的状态是有穷的,因此存在状态集合$S$使得每个状态$s\in S$都在$\sigma$中无穷次出现,而且对于每个公式$f\in C$都存在$S$中的某个状态使得$f$成立。否则,如果对于任意状态集合$S'$中的状态在$\sigma$无穷次出现,都存在公式$f\in C$使得$f$在$S'$中的所有状态上都不成立,那么$f$一定在所有在$\sigma$无穷次出现的状态上都不成立,因此$f$无法在$\sigma$中无穷次成立。假设$S = \{s_{i_1},s_{i_2},...,s_{i_k}\}$,而且$i_1\le i_2\le ...\le i_k$,那么令$s'_p = s_{i_1}$,$s'_n = s_{i_{k'}}$使得$i_{k'} \ge i_k$以及 $s_{i_{k'}} = s_{i_1}$.
\end{proof}

\begin{proposition}\label{prop:fair_fi}
	令$C$一个\CTLP{}公式的集合,而且$\sigma_f = s_0,s_1,...,s_n$是一条有穷路径,其中对任意$0\le i\le n-1$都有$s_i\rightarrow s_{i+1}$,而且存在$0\le p\le n$使得$s_p = s_n$,而且对任意$C$中的公式,都存在$s_p$和$s_n$之间的某个状态使得该公式在该状态上成立。那么一定存在一条无穷路径$\sigma = s'_0, s'_1,...$使得对任意$i\ge 0$都有$s'_i \rightarrow s'_{i+1}$,而且每个状态$s'_j$都在$s_0,s_1,...,s_n$中出现,而且$C$中的每个公式都在该无穷路径上无穷次成立。
\end{proposition}
\begin{proof}
	令$\sigma = s_0,...,s_{p-1},s_p,...,s_{n-1},...$ 
	即可得证。
\end{proof}

\section{本章小节}


