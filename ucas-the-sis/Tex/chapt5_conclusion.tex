\chapter{论文总结与对未来工作的展望}\label{chapt:conclusion}
本论文的内容是从定理证明的角度来研究模型检测问题。具体方式为:首先,在逻辑层面对模型检测中常用的计算树逻辑\CTL{}进行扩展,我们称扩展后的时序逻辑为\CTLP{}。相比于\CTL{},\CTLP{}可表达多元谓词。多元谓词的存在使得扩展后的逻辑不仅可以表达状态的时序性质,同时可以表达状态之间的“空间”性质,即多个状态之间的关系。然后,我们提出了一种针对\CTLP{}逻辑的证明系统\SCTL{}。\SCTL{}以一个Kripke模型为参数,一个\CTLP{}公式在\SCTL$(\cal M)$是可证的当且仅当该公式在当前的Kripke模型$\cal M$中是成立的。然后,我们对于\SCTL{}证明系统设计了一种高效的证明搜索算法,并基于此算法开发了一个定理证明工具\sctlprov{}。\sctlprov{}既可以看作为定理证明工具,也可看作为模型检测工具。相比于传统的定理证明工具,\sctlprov{}可利用一些模型检测中的常用策略来提高验证的效率;相比于传统的模型检测工具,\sctlprov{}的输出结果更加丰富。最后,为了更好的理解\sctlprov{}的验证结果,我们开发了一个定理证明可视化工具\tool{VMDV},同时,为了适配更多的定理证明工具,\tool{VMDV}提供了一个一般化的接口用来与不同的定理证明工具通讯,从而可视化不同的定理证明工具的输出结果。

基于本论文的工作,未来我们还可在以下几个方面对已有的工作进行扩展:第一,我们可将\CTLP{}逻辑以及\SCTL{}证明系统扩展到基于符号迁移系统的验证中,而不仅仅是针对Kripke模型的验证。第二,我们可将\sctlprov{}扩展到多线程版本,并利用现代计算机具有多个核心的特性来进一步提高验证效率。第三,我们还可继续在\tool{VMDV}方面的工作,实现针对更多更复杂的定理证明工具的证明结果以及证明过程的可视化。