\chapter{\tool{Coq}的证明可视化}\label{chapt_coqv}
在上一章中我们提到,\tool{VMDV}提供了一个一般化的接口,用于与不同的定理证明工具进行通讯,进而可以实现不同定理证明工具的证明可视化。本章介绍的是对于一个成熟的,并且在工业界以及学术界得到广泛使用的定理证明工具\tool{Coq}的证明可视化。

\section{\tool{Coq}概述}
\couic{The Coq system is designed to develop mathematical proofs, and especially to write formal specifications, programs and to verify that programs are correct with respect to their specification. It provides a specification language named Gallina. Terms of Gallina can represent programs as well as properties of these programs and proofs of these properties. Using the so-called Curry-Howard isomorphism, programs, properties and proofs are formalized in the same language called Calculus of Inductive Constructions, that is a λ
	-calculus with a rich type system. All logical judgments in Coq are typing judgments. The very heart of the Coq system is the type-checking algorithm that checks the correctness of proofs, in other words that checks that a program complies to its specification. Coq also provides an interactive proof assistant to build proofs using specific programs called tactics.

Coq has an interactive mode in which commands are interpreted as the user types them in from the keyboard and a compiled mode where commands are processed from a file.

The interactive mode may be used as a debugging mode in which the user can develop his theories and proofs step by step, backtracking if needed and so on. The interactive mode is run with the coqtop command from the operating system (which we shall assume to be some variety of UNIX in the rest of this document).
The compiled mode acts as a proof checker taking a file containing a whole development in order to ensure its correctness. Moreover, Coq’s compiler provides an output file containing a compact representation of its input. The compiled mode is run with the coqc command from the operating system.
}
\tool{Coq}是一个利用人工辅助来开发数学证明的工具,尤其是被用来编写程序和程序规范,以及验证程序是否符合规范。\tool{Coq}的脚本语言被称为Gallina。Gallina中的项(Term)可用来表示程序以及程序规范。根据柯里-霍华德同构,程序、规范以及证明都可以用同一种语言来定义,即归纳构造演算(Calculus of Inductive Construction)\cite{BertotC04}。在这个语言中,所有的逻辑判断(logical judgement)都是类型判断(type judgement),而\tool{Coq}的核心功能即利用类型检查算法来检验证明的正确性,即检查一段程序是否符合其规范。除此之外,\tool{Coq}还提供了一种被称为策略(tactic)的机制来构造证明。

\tool{Coq}提供了两种构造证明的模式:交互模式和编译模式。
\begin{itemize}
	\item 交互模式:用户通过在操作系统的命令行中输入“coqtop”来开启交互模式。
	在这种模式中,用户可以通过逐步输入命令的方式来构造理论和证明。
	\item 编译模式:用户通过在操作系统的命令行中输入“coqc”来调用\tool{Coq}的编译模块。编译模块可被看作为证明检查工具,并来确保所输入的文件中包含的脚本代码是正确的。
\end{itemize}

\subsection{\tool{Coq}的用户接口}
\tool{Coq}的用户接口的作用是辅助用户编写证明脚本,并实时显示\tool{Coq}对证明脚本解析状态,反过来帮助用户完成余下的证明脚本的编写。目前最常用的\tool{Coq}用户接口为\tool{Proof General}\footnote{\url{https://proofgeneral.github.io}}和\tool{CoqIDE}\footnote{\url{https://github.com/coq/coq/wiki/CoqIde}}。\tool{Proof General}是一个基于Emacs\footnote{\url{https://www.gnu.org/software/emacs/}}的定理证明工具的前端(front end),并提供了一个统一的接口来适配不同的定理证明工具。不同于\tool{Proof General},\tool{CoqIDE}是一个独立的,专门针对\tool{Coq}的图形用户界面系统。
在这两个用户接口程序中都可实现\tool{Coq}的交互和编译两种模式的证明。

\subsection{\tool{Coq}文档的内部表示}




\section{\tool{Coq}中证明树的构造与可视化}
\tool{Coq}中的证明可以展现成证明树的形式:每个当前需要证明的命题被称为目标(goal),当应用一条规则,而且该命题匹配当前规则的结论,那么利用与命题和规则的结论同样的匹配方式去例化当前规则的假设,从而得到0个或多个命题,这些新生成的命题被成为子目标(subgoal),然后在新生成的命题上利用相同的证明方式去产生更多的子目标,直到无法生成更多的子目标为止。在这种形式的证明中,每个目标可看作证明树中的一个节点,而其子目标则可看作该节点的子节点。在\tool{Coq}中对一个命题的证明过程则可看作一棵证明树的构造过程,在此过程中所应用的规则即证明脚本中的策略。

然而,无论在\tool{Coq}中,还是在其用户接口\tool{Proof General}和\tool{CoqIDE}中,都不存在构造以及保存证明树的机制,因此这种状况给用户在证明过程中增添了一种无形的负担,即需要自行在脑海中想象当前证明树的完整形状。随着证明操作越来越复杂,比如需要回退操作以及在不同子目标之间相互跳转时等,用户所依赖的信息只有\tool{Coq}给出的当前需要证明的命题的信息,而越来越难以建立起完整的证明树的概念。
为了解决这个问题,我们在本章中研究\tool{Coq}的证明可视化,因此我们设计并开发了一个工具\tool{Coqv}\footnote{\url{https://github.com/ProveVis/coqv}}来完整地记录\tool{Coq}中证明树以及对证明树的操作。\tool{Coqv}通过与\tool{VMDV}结合,就可实现\tool{Coq}的证明树的可视化。

\subsection{\tool{Coqv}的原理}
\tool{Coqv}中构造证明树的原理可总结为:
\begin{enumerate}
	\item 首先,\tool{Coqv}接收用户输入的命令,然后将该命令发送到\tool{Coqtop};
	\item 然后,\tool{Coqv}等待\tool{Coqtop}发出的反馈消息;
	\item 最后,\tool{Coqv}解析所收到的消息,然后根据解析的消息内容更新证明树。
\end{enumerate}

在第1步和第2步中,\tool{Coqv}与\tool{Coqtop}的是通过\tool{Coqtop}内置的一个XML协议\footnote{\url{https://github.com/siegebell/vscoq/wiki/XML-protocol}}来通讯的。
