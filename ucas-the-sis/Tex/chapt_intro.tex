\chapter{引言}\label{chapt:intro}
\section{选题背景和意义}
近年来,计算机软硬件系统的飞速发展极大地提高了人类的生产效率,也极大地方便了人类的生活。但与此同时,由于软硬件故障而导致的人身以及生产事故同样不可忽视,
尤其是对于安全攸关的系统,比如航空航天系统、高速铁路控制系统、医疗卫生系统等,这些系统的故障往往会导致灾难性的后果。比如,1994年,Intel公司由于其发售的奔腾系列处理器的指令集出现漏洞而宣布召回所有缺陷产品,并因此造成4.75亿美元的经济损失。1996年,由于飞行控制程序中存在由64位浮点数到16位整数的错误转换,阿丽亚娜5号运载火箭在首次发射中仅仅飞行37秒就发生爆炸,并造成有史以来损失最惨重的航天事故。因此,随着软硬件系统的复杂度的提高,如何确保系统的正确性成为系统开发面临的关键性问题之一。

传统的计算机软硬件系统的开发一般依靠仿真和测试的方法来评估系统是否满足设计要求。基于仿真和测试的方法在系统开发的初始阶段可以很大地提高开发效率,并便于即时调试。随着技术的不断发展和完善,此类方法已经可以越来越方便地发现系统设计的一些隐匿的错误。由于其高效性,现如今的大量软硬件系统的开发过程依然采用此类方法。然而,基于仿真和测试的方法所能覆盖的系统行为有限,往往无法确保找到系统的所有错误,因而无法保证系统的正确性。

除了基于仿真和测试的方法,对系统正确性的评估还可通过形式化验证来进行。与基于仿真和测试的方法不同,通过形式化方法验证过的系统通常可确保其正确性。接下来,我们详细介绍形式化方法。
\section{形式化方法}

%\subsection{什么是形式化方法?}
形式化方法指的是一系列基于严格的数学理论,并以数学证明来确定计算机软硬件系统的正确性的方法。
%形式化方法的应用通常贯穿于系统的描述、开发和验证过程。
%在利用形式化方法验证系统性质的过程中,系统的所有状态或行为通常用形式化描述语言来定义,然后分别利用数学的方法枚举由形式语言所定义的所有的系统状态,直到得出验证结论。
形式化方法的应用通常包含以下四部分:
\begin{enumerate}
	\item 在一个具有严格数学定义的语义的形式化语言中描述需要验证的系统;
	\item 形式化定义系统的性质(规范);
	\item 利用算法来验证系统是否满足规范;
	\item 分析验证结果。
\end{enumerate}
在众多的形式化验证方法中,模型检测和定理证明是最被普遍应用的两种。
\paragraph{模型检测}
20世纪80年代,模型检测\cite{CGP01,BouajjaniJNT00,BaierKatoen08}最早由Clark和Emerson等人提出,并很快得到迅猛发展。
在模型检测中,被验证的系统通常被抽象为一个有限状态模型,模型中的有限个状态对应于系统的状态,同时模型中的状态迁移通常用来描述系统在运行过程中的状态变化。系统的性质通常由时序逻辑公式来描述。
根据使用场景与表达能力的不同,常用的时序逻辑可分为\textsf{CTL}$^*$\cite{EmersonH86}、\textsf{CTL}\cite{ClarkeE08}、\textsf{LTL}\cite{Pnueli77}、Hennessy-Milner逻辑\cite{HennessyM85}、\textsf{ACTL}$^*$\cite{NicolaV90}等。其中\textsf{CTL}$^*$、\textsf{CTL}、\textsf{LTL}被用来描述Kripke模型的性质,而Hennessy-Milner逻辑、\textsf{ACTL}$^*$被用来描述标号迁移系统的性质。
在模型检测中,验证系统的性质通常指的是判定时序逻辑公式在有限状态模型上的可满足性。模型检测工具通常会通过遍历状态空间来验证公式的可满足性。当需要验证的公式是可满足的时候,模型检测工具会返回true;反之,当需要验证的公式不可满足时,模型检测工具会产生一个反例用来定位系统的错误。近年来,随着技术和理论的不断发展,一大批优秀的模型检测工具不断地涌现出来,其中被广泛应用的模型检测工具以符号模型检测工具\nusmv{}\cite{CimattiCGR99}、限界模型检测工具\verds{}\cite{Zhang14}、即时模型检测工具\tool{SPIN}\cite{Holzmann97}以及形式化验证工具集\CADP{}\cite{GaravelLMS13}为代表。
\begin{itemize}
	\item \nusmv{}:\nusmv{}的提出是基于最早由McMillan在其博士论文\cite{mcmillan93}中提出的符号模型检测方法。符号模型检测主要应用于对计算树逻辑(Computation Tree Logic,简称\CTL{})公式的验证,符号模型检测最大的特点在于将要验证的性质看成系统状态的集合,而状态集合通常用\BDD{}\cite{Bryant86}来进行表示。一个\CTL{}公式的可满足行则表示为系统的初始状态是否属于该状态集合。在系统的验证过程中,系统的状态空间不是显示地生成出来,而是由\BDD{}来符号化地表示,这种状态集合的表示方式在表达相同数量的状态时相比显示表示每个状态的方式通常占用更少的空间。
	\item \verds{}:
	限界模型检测工具\verds{}由Zhang\cite{Zhang14}提出,不同于符号模型检测工具,\verds{}在验证公式的过程中首先规定一个界,然后在界内搜索模型的状态。在实际的验证过程中,相比于\nusmv{},\verds{}通常只需访问较少的状态就能完成公式的验证。另外,不同于基于\textsf{SAT}\cite{BCCZ99}的限界模型检测工具,\verds{}在验证过程中将时序逻辑公式翻译到一个\textsf{QBF}(Quantified Boolean Formula),相比于\textsf{SAT}公式,\textsf{QBF}可用来表示更加复杂的时序逻辑公式。
	\item \tool{SPIN}:\tool{SPIN}是在1980年由美国贝尔实验室的Holzmann等人编写,并在1991年免费开放的一种即时模型检测工具。不同于\nusmv{}和\verds{},\tool{SPIN}更多的时候被用来验证分布式系统模型的线性时序逻辑(Linear Temporal Logic,简称\textsf{LTL})性质。在\tool{SPIN}的验证过程不需要提前规定模型搜索的界,而是通过深度优先的方式逐个访问模型的状态,直到可以计算出\textsf{LTL}公式的可满足性为止。另外,不同于\nusmv{}和\verds{},模型的状态在\tool{SPIN}中通常显式存储,不过由于需要访问的状态通常比前两种工具少,因此在某些模型的验证中空间占用比前两种工具更少。
	\item \tool{CADP}:\tool{CADP}全称Construction and Analysis of Distributed Processes,是一系列用于通讯协议和分布式系统设计和验证的工具集合。\tool{CADP}目前由法国国家信息与自动化研究所的\textsf{CONVECS}小组开发(起初由\textsf{VASY}小组开发),并在工业界得到广泛应用。不同于验证Kripke模型的工具,\textsf{CADP}主要用来验证基于动作的形式化系统,比如符号迁移系统、马尔可夫链等。\CADP{}工具集中包含可验证多种时序逻辑公式的工具,并可采用即时模型检测或符号模型检测的状态搜索策略。
\end{itemize}



\paragraph{定理证明}
不同于基于状态的模型检测方法,定理证明\cite{Fitting96,Loveland78,Burel09}是基于公理与推理规则的。在利用定理证明方法形式化验证系统的性质时,被验证的系统通常在一种数学语言中被表述为一系列数学定义,而要验证的性质则被表述为数学逻辑公式,然后从公理出发,利用推理规则寻找逻辑公式的形式化证明。在应用定理证明方法进行形式化验证时,定理证明工具的产生极大地增强了证明的效率。根据基于的数学逻辑的不同,定理证明工具可大体被分为以下几类:
\begin{itemize}

	\item \textbf{经典高阶逻辑}:经典高阶逻辑的表达能力较强,通常可方便地用来描述复杂的系统行为和性质。然而,较好的表达性也意味着验证的自动化程度较低。基于经典高阶逻辑的代表性定理证明工具有\tool{HOL}\cite{Gordon00}、\tool{Isabelle/HOL}\cite{Nipkow12}、\tool{PVS}\cite{OwreRS92}等。
	\item \textbf{构造逻辑}:由柯里-霍华德同构\cite{PoernomoWC05}可知,一个公式的构造性证明一一对应于一个可执行的计算机算法。因此,在自动化程度上,基于构造逻辑的定理证明工具相比基于经典高阶逻辑的工具更好。但是,公式的构造性证明往往比较难以找到。基于构造逻辑的代表性定理证明工具有\tool{Coq}\cite{BertotC04}、\tool{Nuprl}\cite{AllenCEKL00}等。
	\item \textbf{一阶逻辑}:在表达能力上,基于一阶逻辑的定理证明工具通常可以完成绝大多数系统的验证,同时此类定理证明器的自动化程度较高。基于一阶逻辑的代表性定理证明工具有\tool{ACL2}\cite{KaufmannM08}、\tool{Isabelle/FOL}\cite{cs-LO-9301106}等。
\end{itemize}

\paragraph{模型检测 vs. 定理证明}
作为两种主要的形式化验证方法,模型检测和定理证明的优缺点是互补的。
相比于定理证明,模型检测的主要优点是验证过程完全自动化,主要缺点是状态爆炸问题,尤其是在对多进程的并发系统的验证过程中,随着进程数量的增长,系统状态的数量可能呈现指数级的增长。状态爆炸问题是困扰模型检测在大型系统的验证中应用的主要问题。
相比于模型检测,定理证明的优点在于表达能力强,而且验证过程不是基于状态的,因此可以被用来验证大型系统,甚至是无穷状态系统。不过,相比于模型检测,定理证明的缺点就是证明过程通常需要人工干预,而不是完全自动化的。
除了验证过程,定理证明工具的验证结果也与模型检测方法不同:定理证明的验证结果一般是以证明树的形式给出,比模型检测的验证结果包含更详细的信息。



\section{相关工作}
上文中提到,模型检测和定理证明的优缺点是互补的,而如何将这两种形式化验证方法相结合成为近年来的热门研究课题,这也是本论文的主要研究内容。截至目前,国际上已经有多个试图结合模型检测和定理证明两种方法的工作。
\begin{itemize}
	\item Rajan,Shankar和Srivas提出了一种基于\tool{PVS}的模型检测和定理证明的结合方法\cite{RajanSS95}。在这种方法中,作者首先在\tool{PVS}中定义了一个$\mu$-演算理论,并用$\mu$-演算定义了时序逻辑\textsf{CTL}的模态词,当证明\textsf{CTL}公式的时候,\tool{PVS}则会调用调用内置的模型检测命令来完成证明。在这种方法中,模型检测方法被用来完成\tool{PVS}证明的子目标,而不是贯穿于整个证明过程。这种将模型检测和定理证明相结合方式的最大优点是利用了定理证明器强大的表达和抽象能力将系统抽象成一个有穷状态模型,然后利用模型检测算法在有穷状态模型中完成状态搜索。
	\item 在证明\textsf{PLTL}公式方面,Cavalli和Cerro给出了一种基于经典的resolution的方法\cite{CavalliC84}。在这种方法中,作者将\textsf{PLTL}公式翻译到一种时序逻辑的合取范式,然后定义了一组这种合取范式的resolution规则。与这种方法类似,Venkatesh提出了一种基于resolution的判定\textsf{LTL}公式可满足性的方法\cite{Venkatesh85}。这种方法的特点是将给定的\textsf{LTL}公式转换成与该公式的可满足性等价的带有嵌套模态词的公式,然后在转换后的公式上应用经典的resolution规则。同样基于resolution的方法还有Fisher,Dixon和Peim提出的证明\textsf{LTL}公式有效性的方法\cite{FisherDP01},以及Zhang,Hustadt和Dixon提出的证明\textsf{CTL}公式有效性的方法\cite{ZhangHD14}等。
	\item Ji提出了一种基于resolution modulo的用于验证\textsf{CTL}公式有效性的方法\cite{Ji15}。Resolution modulo方法首次由Dowek,Hardin和Kirchner提出\cite{DowekHK03,Dowek10},基本思想是将一阶逻辑的resolution方法转换成重写系统,因此公式的推导过程更多地用计算(重写)来实现。Ji的方法已经在定理证明器\tool{iProver Modulo}中实现,并在与现有的模型检测工具(\nusmv{}和\verds{})的实验结果对比中表现良好。
\end{itemize}
\section{本文的内容和组织结构}
本文的主要内容是研究模型检测和定理证明两种方法的结合。不同于已有的工作,在理论方面,本文既没有将模型检测作为定理证明的一个判定过程,也没有将已有时序逻辑公式的模型检测问题转化为定理证明问题,而是提出基于传统时序逻辑的一个扩展,并在保持模型检测完全自动化优点的基础上可以验证更复杂的性质。
在工具实现方面,我们同时结合了模型检测工具与定理证明工具输出结果的特点:既保留了模型检测工具对于状态细节的输出,也遵循了定理证明工具的风格将公式的完整的证明树输出。这种形式的验证结果输出同时也带来了一个问题,即证明输出的结构往往被过多的细节所掩盖,尤其是对于大型系统的验证而言。为了解决这个问题,我们提出了证明可视化的概念,通过可视化分析输出结果可以很好地解决这个问题。
本论文内容最初基于蒋颖教授与Gilles Dowek教授的工作\cite{GY13}。本文组织结构如下:
\begin{enumerate}
	\item 建立一个针对\CTL{}逻辑的扩展:\CTLP{}。相比于\CTL{},在\CTLP{}中可以定义多元谓词。除了可以定义状态的时序性质之外,利用多元谓词可以表达某些“空间”性质,即多个状态之间的关系。针对\CTLP{},我们还定义了一个证明系统\SCTL{}用来表达\CTLP{}公式的证明,并证明了该证明系统的可靠性和完备性。除此之外,我们还提出了一种针对\SCTL{}系统的证明搜索算法并证明了该算法的终止性和正确性。
	\item 给出了一个对证明系统\SCTL{}的编程实现---定理证明工具\sctlprov{}。在不丢失效率的前提下,\sctlprov{}可被用来实现复杂系统的验证,\sctlprov{}的输出结果相比于传统的模型检测工具更丰富。为了对比\sctlprov{}与现有工具的效率,我们分别在若干个随机生成以及工业级测试用例集上与多个业界顶尖的形式化验证工具进行实验数据的对比。
	\item 设计并实现了定理证明可视化工具\tool{VMDV},并将\tool{VMDV}用于\sctlprov{}的证明输出的可视化。同时\tool{VMDV}还被设计并实现为一般化的证明可视化工具,并可实现不同的定理证明工具输出的可视化。
	\item 作为一个扩展性工作,实现了定理证明工具\tool{Coq}中构造证明树以及可视化证明的方法。
\end{enumerate}

除本章外,本文其余章节的组织结构如下:

第\ref{chapt:preliminary}为预备知识。

第\ref{chapt:sctl}章介绍了\CTLP{}逻辑和证明系统\SCTL{},以及针对\SCTL{}的证明搜索算法。证明了\SCTL{}证明系统的可靠性和完备性,以及证明搜索算法的终止性和正确性。证明搜索算法的伪代码也在本章给出。

%第\ref{chapt:fairness}章介绍了公平性性质的定义,以及针对公平性性质的验证方法。

第\ref{chapt:sctlprov}章介绍了定理证明工具\sctlprov{}的架构和实现细节,并在多个测试用例集上与多个形式化验证工具进行实验结果对比。同时给出\sctlprov{}验证复杂系统(模型检测工具无法处理)的一个案例的分析。

第\ref{chapt:visulization}章介绍了定理证明可视化工具\tool{VMDV}的架构和实现细节。

第\ref{chapt:coqv}章介绍的是定理证明可视化工具\tool{VMDV}在定理证明工具\tool{Coq}上的应用。

第\ref{chapt:conclusion}章是全文总结和对未来工作的展望。