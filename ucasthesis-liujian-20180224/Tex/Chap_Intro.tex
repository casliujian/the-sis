
\chapter{引言}
\label{chap:introduction}

考虑到大多数用户并无\LaTeX{}使用经验,本模板将\LaTeX{}的复杂性尽可能地进行了封装,开放出简单的接口,以便于使用者可以轻易地使用。同时,对使用\LaTeX{}撰写论文所遇到的一些主要难题,如插入图片、文献索引等,进行了详细的说明,并提供了相应的代码样本,理解了上述问题后,对于初学者而言,使用此模板撰写其学文论文将不存在实质性的困难。所以,如果您是初学者,请不要直接放弃,因为同样作为初学者的我,十分明白让\LaTeX{}变得简单易用的重要性,而这正是本模板所体现的。

此中国科学院大学学位论文模板\texttt{ucasthesis}基于吴凌云的\texttt{CASthesis}模板发展而来,ucasthesis文档类的基础架构为ctexbook文档类。当前ucasthesis 模板满足最新的中国科学院大学学位论文撰写要求和封面设定。模板兼顾不同操作系统 (Windows, Linux, Mac OS) 并兼容 pdflatex 和 xelatex 编译方式,完美地支持中文书签、中文渲染、中文粗体显示、拷贝pdf中的文本到其他文本编辑器等特性,此外,对模板的文档结构进行了精心设计,撰写了编译脚本提高模板的易用性和使用效率。

宏包的目的是简化学位论文的撰写,模板文档的默认设定是十分规范的,从而论文作者可以将精力集中到论文的内容上,而不需要在版面设置上花费精力。 同时,在编写模板的\LaTeX{}文档代码过程中,作者对各结构和命令进行了十分详细的注解,并提供了整洁一致的代码结构,对文档的仔细阅读可以为初学的您提供一个学习\LaTeX{}的窗口。除此之外,整个模板的架构十分注重通用性,事实上,本模板不仅是中国科学院大学学文论文模板,同时,也是使用\LaTeX{}撰写中英文article或book的通用模板,并为使用者的个性化设定提供了接口和相应的代码。

\section{系统要求}

\texttt{ucasthesis}宏包可以在目前大多数的\TeX{}编译系统中使用,例如C\TeX{}、MiK\TeX{}、\TeX{}Live。推荐的\TeX{}编译系统 + 文本编辑器为
\begin{itemize}
    \item Linux: \TeX{}Live + vim or Texmaker
    \item MacOS: \TeX{}Live or Mac\TeX{} + Macvim or Texmaker
    \item Windows: \TeX{}Live or Mik\TeX{}  + Texmaker
\end{itemize}
\TeX{}编译系统 (如MiK\TeX{}、\TeX{}Live) 用于提供编译环境,文本编辑器 (如Texmaker、vim) 用于编辑\TeX{}源文件。

\section{问题反馈}

\begin{center}
莫晃锐 (mohuangrui) \quad mohuangrui@gmail.com

模版下载地址: \url{https://github.com/mohuangrui/ucasthesis}
\end{center}

欢迎大家反馈模板不足之处,一起不断改进模板。希望大家向同事积极推广\LaTeX{},一起更高效地做科研。
